\documentclass[11pt]{article}
\pagestyle{myheadings}
\markright{Utilisation d’Ordinateurs en réseau} %nom de l’en-tête

\usepackage{geometry}
\usepackage[french]{babel}
\usepackage[T1]{fontenc}
\usepackage{array}
\usepackage{graphicx}
\usepackage{xcolor}
\usepackage[colorlinks=true, urlcolor=blue, linkcolor=blue]{hyperref}

\usepackage[most]{tcolorbox}
\tcbuselibrary{listings}

\usepackage{tabularx}
\usepackage{multirow,bigdelim}
\usepackage{amsmath}
% Pour aligner à droite une image dans figure
\usepackage[export]{adjustbox}


%%% Pour numéroter les sections avec des lettres
\usepackage{engrec}
\renewcommand{\thesection}{\Alph{section}}


%%%%%%%%%%%CODE%%%%%%%%%%%%%
%Ceci résoud le problème des caractères unicode, éhonteusement pompé sur le modèle d’Alexandre Martos
% Default options
\lstset{
    % showspaces=true, % debug purposes
    sensitive=true,
    showstringspaces=false,
    columns=fullflexible, % no additional spaces due to font
    upquote=true,
    keepspaces=true,
    breaklines=true,
    breakatwhitespace=true,
    belowskip=0.5cm,
    extendedchars=true,
    % unicode characters not properly handled. See
    % https://en.wikibooks.org/wiki/LaTeX/Source_Code_Listings#Encoding_issue
    literate= % ONLY ONE COMA FOR ALL
      {á}{{\'a}}1 {é}{{\'e}}1 {í}{{\'i}}1 {ó}{{\'o}}1 {ú}{{\'u}}1 % chktex 7
      {Á}{{\'A}}1 {É}{{\'E}}1 {Í}{{\'I}}1 {Ó}{{\'O}}1 {Ú}{{\'U}}1
      {à}{{\`a}}1 {è}{{\`e}}1 {ì}{{\`i}}1 {ò}{{\`o}}1 {ù}{{\`u}}1 % chktex 7
      {À}{{\`A}}1 {È}{{\'E}}1 {Ì}{{\`I}}1 {Ò}{{\`O}}1 {Ù}{{\`U}}1
      {ä}{{\"a}}1 {ë}{{\"e}}1 {ï}{{\"i}}1 {ö}{{\"o}}1 {ü}{{\"u}}1 % chktex 7
      {Ä}{{\"A}}1 {Ë}{{\"E}}1 {Ï}{{\"I}}1 {Ö}{{\"O}}1 {Ü}{{\"U}}1
      {â}{{\^a}}1 {ê}{{\^e}}1 {î}{{\^i}}1 {ô}{{\^o}}1 {û}{{\^u}}1 % chktex 7
      {Â}{{\^A}}1 {Ê}{{\^E}}1 {Î}{{\^I}}1 {Ô}{{\^O}}1 {Û}{{\^U}}1
      {Ã}{{\~A}}1 {ã}{{\~a}}1 {Õ}{{\~O}}1 {õ}{{\~o}}1
      {œ}{{\oe}}1 {Œ}{{\OE}}1 {æ}{{\ae}}1 {Æ}{{\AE}}1 {ß}{{\ss}}1
      {ű}{{\H{u}}}1 {Ű}{{\H{U}}}1 {ő}{{\H{o}}}1 {Ő}{{\H{O}}}1
      {ç}{{\c c}}1 {Ç}{{\c C}}1 {ø}{{\o}}1 {å}{{\r a}}1 {Å}{{\r A}}1
      {€}{{\euro}}1 {£}{{\pounds}}1 {«}{{\guillemotleft}}1
      {»}{{\guillemotright}}1 {ñ}{{\~n}}1 {Ñ}{{\~N}}1 {¿}{{?`}}1,
      % {-}{-}1, % css needs dashes
      % adding the star mess up syntax color in c source files
      % {*}{*}1,
}

%%%%%%%%%%%%%%%%%%%%%%%%%%%%%%%%%%%%%%%%%%%%%%%%%%%%%%%%%%%%%%%%%%%%%%%%%%%%%%%
%%%%%%%%%%%%%%%%%%% Ajout pour blocs code plus lisibles %%%%%%%%%%%%%%%%%%%%%%%

%%%%%%%%%%%%%%%%%%%%%%%%%%%%%%%%%%%%%%%%%%%%%%%%%%%%%%%%%%%%%%%%%%%%%%%%%%%%%%%
% Syntax colors
%%%%%%%%%%%%%%%%%%%%%%%%%%%%%%%%%%%%%%%%%%%%%%%%%%%%%%%%%%%%%%%%%%%%%%%%%%%%%%%

\colorlet{lightgrey}{black!5}

% Parrots colors
% see https://www.pinterest.fr/pin/534943261958609466/
\definecolor{pteal}{HTML}{1B2D51} % too close to black
\definecolor{pdarkblue}{HTML}{124F8E} % too close to black
\definecolor{pblue}{HTML}{0680CB}
\definecolor{pgreen}{HTML}{639236}
\definecolor{pyellow}{HTML}{F6A609}
\definecolor{pred}{HTML}{F32E25}

% Syntax coloration
\colorlet{background}{lightgrey}
\colorlet{comment}{gray}
\colorlet{string}{pgreen}
\colorlet{keyword1}{pred}
\colorlet{keyword2}{pblue}
\colorlet{keyword3}{pyellow}

%%%%%%%%%%%%%%%%%%%%%%%%%%%%%%%%%%%%%%%%%%%%%%%%%%%%%%%%%%%%%%%%%%%%%%%%%%%%%%%
% Listings
%%%%%%%%%%%%%%%%%%%%%%%%%%%%%%%%%%%%%%%%%%%%%%%%%%%%%%%%%%%%%%%%%%%%%%%%%%%%%%%

% do not put curly double quotes. hack, but no choice.
% see https://tex.stackexchange.com/a/566480

% Font style
\lstdefinestyle{fnttstyle}{basicstyle=\footnotesize\ttfamily}

% syntax coloration
\lstdefinestyle{nosyntaxcol}{keywordstyle=\normalsize} % disable
\lstdefinestyle{syntaxcol}{ % enable
    keywordstyle=\color{keyword1},
    keywordstyle=[2]\color{keyword2},
    keywordstyle=[3]\color{keyword3},
    stringstyle=\color{string},
    commentstyle=\color{comment},
}

% Définition du style principal des codes sources
% styles définis dans src/codeblocks.tex
\lstdefinestyle{CodeBox}{
    style=tab4, % la largeur de tabulation vaut 4 espaces
    style=linesnb5, % numéro de lignes, toutes les 5 lignes
    style=box, % enferme les codes sources dans des boites colorées
    style=fnttstyle, % police type typewriter
    style=syntaxcol, % coloration syntaxique
}

%%%%%%%%%%%%%%%%%%%%%%%%%%%%%%%%%%%%%%%%%%%%%%%%%%%%%%%%%%%%%%%%%%%%%%%%%%%%%%%
% tcolor boxes
%%%%%%%%%%%%%%%%%%%%%%%%%%%%%%%%%%%%%%%%%%%%%%%%%%%%%%%%%%%%%%%%%%%%%%%%%%%%%%%

\newtcolorbox{graybox}{
    boxrule=0pt,
    breakable,
    colback=gray!10,
    width=1.07\textwidth,
}

% indentation
\lstdefinestyle{tab2}{tabsize=2} % 2 spaces
\lstdefinestyle{tab4}{tabsize=4} % 4 spaces
\lstdefinestyle{tab8}{tabsize=8} % 8 spaces

% line numbers
\lstdefinestyle{linesnbdefault}{
    numbers=left,
    numbersep=10pt, % distance from code
    firstnumber=1,
    numberfirstline=true,
    numberstyle=\tiny\ttfamily,
}

\lstdefinestyle{linesnb5}{
    style=linesnbdefault,
    stepnumber=5 % every 5 lines
}

% code box
\lstdefinestyle{box}{
    backgroundcolor=\color{background},
    rulecolor=\color{gray},
    frame=trBL,
}
%%%%%%%%%%%%%%%%%%%%%%%%%%%%% Fin ajout %%%%%%%%%%%%%%%%%%%%%%%%%%%%%%%%%%%%%%
%%%%%%%%%%%%%%%%%%%%%%%%%%%%%%%%%%%%%%%%%%%%%%%%%%%%%%%%%%%%%%%%%%%%%%%%%%%%%%

%%%%%%Ceci définit le style d’écriture pour du code, cf FAQ Gilles Bernard%%%%
\newtcblisting{code2}{%
colback=lightgray,
boxrule=-1pt,
listing only,
top=-1pt,
bottom=-2pt,
left=-1pt,
right=0pt,
boxsep=0pt,
box align=center,
after={\par\smallskip\noindent\phantom{i}},
 listing options={
    language=HTML,
    backgroundcolor=\color{lightgray},
    basicstyle=\sffamily\normalsize,
    keywordstyle=\sffamily\normalsize,
    upquote=true,
    showstringspaces=false,
    showspaces=false,
    tabsize=2,
    xleftmargin=15pt,
    breaklines=true,
    breakatwhitespace=true,
    columns=fullflexible,
    commentstyle=\sffamily\itshape,
    escapeinside={\%/*}{*/},
  }
}

\newtcblisting{code3}{%
colback=lightgray,
boxrule=-1pt,
listing only,
top=-1pt,
bottom=-2pt,
left=-1pt,
right=0pt,
boxsep=0pt,
box align=center,
after={\par\smallskip\noindent\phantom{i}},
 listing options={
    language=PHP,
    backgroundcolor=\color{lightgray},
    basicstyle=\sffamily\normalsize,
    keywordstyle=\sffamily\normalsize,
    upquote=true,
    showstringspaces=false,
    showspaces=false,
    tabsize=2,
    xleftmargin=15pt,
    breaklines=true,
    breakatwhitespace=true,
    columns=fullflexible,
    commentstyle=\sffamily\itshape,
    escapeinside={\%/*}{*/},
  }
}

\makeatletter
% tirets
\lst@CCPutMacro
   \lst@ProcessOther {"2D}{\lst@ttfamily{-{}}{-}}
    \@empty\z@\@empty
% guillemets
\lst@CCPutMacro
   \lst@ProcessOther {"22}{\lst@ifupquote \textquotedbl%
     \else \char34\relax \fi}
    \@empty\z@\@empty
\makeatother

%%%%%%%%%%%%%%%%%%%%%% Deuxième ajout %%%%%%%%%%%%%%%%%%%%%%%%%%%%%
\lstset{style=CodeBox}
%%%%%%%%%%%%%%%%%%%%%% Fin deuxième ajout %%%%%%%%%%%%%%%%%%%%%%%%%

%%%%%%%%%%%%%%%%%%%%%% FIN % CODE %%%%%%%%%%%%%%%%%%%%%%%%%%%%%%%%%

\geometry{margin=70pt}


\begin{document}

%%%%%%%%PAGE%DE%GARDE%%%%%%%%%%%%%
\title{Informatique Fondamentale - Partie 2} % Matière

\newcommand{\name}{Orianne Delmarre} % Prénom Nom
\newcommand{\studentid}{22002015} % n° étudiant
\newcommand{\subtitle}{Utilisation d’Ordinateurs en Réseau} % Titre du devoir
\author{\name{} (n° étudiant: \studentid{})}

\begin{titlepage}

    \begin{center}
        \makeatletter
            {\large \bfseries \@author } \vspace{0.5cm} % chktex 1

            \@title \vspace{0.2cm} % chktex 1

            {\large \bfseries \subtitle{}} \vspace{0.5cm} 

            Le \today
            \vspace{0.5cm}

            \includegraphics[scale=0.1]{logo.png} 
        \makeatother
    \end{center}

\end{titlepage}
%%%%%%%%FIN%PAGE%DE%GARDE%%%%%%%%%%%

\tableofcontents

\newpage

\section*{Chapitre 1 - Introduction générale}
\addcontentsline{toc}{section}{Chapitre 1 - Introduction générale}

	\subsection*{Exercice 1.1}
	\addcontentsline{toc}{subsection}{Exercice 1.1}
	
	Pour ce chapitre, pas d’exercice à réaliser, si ce n’est de :

Trouver un hébergeur(gratuit si possible) afin de pouvoir réaliser en ligne les exercices des chapitres suivants.

Vérifiez que celui-ci vous permette d’accéder à PHP, à au moins une base MySQL (ou équiv) et à son administration (PhpMyAdmin
ou équiv), ainsi qu’au protocole FTP.

Si c’est le cas, inscrivez-vous gratuitement à celui-ci et notez l’URL de votre "première" page dans le PDF.

Pour vous aider : Contactez les étudiants des années passées pour leur demander quel(s) hébergeur(s) ils ont choisi(s). (par ex : https ://fr.000webhost.com, https ://www.netlify.com/, ...)

\medskip

\textbf{Réponse :} J’ai suivi les conseils du cours, et choisi 000webhost.com comme hébergeur. Le site est disponible à l’adresse \url{https://atco.000webhostapp.com/index.html} 

\textbf{Note importante concernant les images :} Latex semble avoir un peu saturé à cause du nombre d’images. Pour une meilleure lisibilité, j’en ai transféré un certain nombre à la fin du rapport, au niveau des annexes (figures >= 10). Même là, elles se baladent où elles veulent, rarement dans la bonne annexe. La référence est le numéro (cliquable) de la figure indiquée.

\section*{Chapitre 2 - Le Langage HTML}
	\addcontentsline{toc}{section}{Chapitre 2 - Le Langage HTML}

	\subsection*{Exercice 2.1 (À Rendre)}
	\addcontentsline{toc}{subsection}{Exercice 2.1 (À Rendre)}
	
	A L’aide d’un éditeur de Textes (Notepad++, Sublime Text (ai la préférence pour celui-ci), Atom (TB également), BBEdit, Coda, Vim, MS Visual Studio Code (TB), TextMate, Emacs, etc.) :
 
Réalisez une page web (à l’ancienne) avec votre CV ou en développant une thématique personnelle (passion(s), loisir(s), sport(s), jeux...).

La page, écrite en HTMLv5 doit contenir au moins du texte (plusieurs paragraphes, avec de vraies phrases : sujet-verbe-complément à minima), des images (>2), des liens (non cassés)(>2).

Les listes, Les tableaux sont encouragés mais facultatifs.

Ajoutez obligatoirement un titre et des métadonnées descriptives complètes au format de votre choix (Dublin Core, Standard Metatags, Open Graph Data, ...) afin d’augmenter la SEO dans l’entête de votre page (balise HEAD).

Essayez d’en soigner la mise en page et placez-la sur le serveur d’hébergement que vous avez choisi (cf Chapitre 1).

Dans le PDF, après avoir rappelé l’énoncé de l’exercice :

\medskip

\textbf{- Indiquez l’URL de votre page si vous ne l’avez pas déjà fait lors de l’exercice 1.1.}

Mon (tout premier) site vous donnera quelques explications sur un métier assez flou pour beaucoup de gens, mais que personnellement je connais bien, puisqu’il s’agit de mon activité principale : celui de contrôleur aérien. Le contenu du site ne fait que survoler (haha) le sujet, en une page, on ne peut guère plus... Vous trouverez la page ne contenant que de l’HTML à l’adresse suivante : \url{https://atco.000webhostapp.com/htmlnu.html}

\medskip

\textbf{- Faites un C.C (Copier-Coller) du code source HTML complet en ajoutant des commentaires sur vos choix et sur ce que vous avez réalisé — ou non réalisé — et/ou des constructions particulières du HTML que vous avez employées.}
	
\medskip

Vous trouverez le code complet de cette page en annexe, ci-dessous les passages les plus notables :

\textbf{En-tête du code HTML} avec métadonnées descriptives : (pour les métadonnées, j’ai suivi les recommandations de \url{developer.mozilla.org/fr/docs/Web/HTML/Element/meta})

\smallskip

	\begin{code2}
<!DOCTYPE html>
<html lang="fr">
  	<head>
    	<title>Contrôleur aérien</title>
    	<meta charset="utf-8">
    	<meta name="description" content="Une première approche(haha) du métier de contrôleur aérien. Réalisé dans le cadre du cours 'Utilisation d'Ordinateurs en Réseau' de P. Kislin">
    	<meta name="url" content="atco.000webhostapp.com/UOR_page1.html">
    	<meta name="author" content="OD">
    	<meta name="keywords" content="contrôle aérien, contrôleur aérien, aiguilleurs du ciel, ICNA, ATCO, navigation aérienne, DGAC">
  	</head>
\end{code2}
	
	\textbf{Début du corps du code HTML :} titre, citation d’introduction (véridique, on m’a aussi déjà demandé si je contrôlais les bagages), barre de navigation du site sous forme de liste ("accueil JS" correspond au chapitre 6, "accueil CSS" au chapitre 3, "accueil HTML" à cette page et "Quizz" au chapitre 5). Puis un premier titre secondaire (h2) avec une introduction générale et une liste de trois éléments pour présenter les missions du contrôleur.
	
\smallskip

	\begin{code2}
  	<body>
    	<h1 title="Un peu plus près des étoiles">Métier : contrôleur aérien</h1>
    	<blockquote>Le contrôleur aérien, c'est celui qui contrôle les billets dans l'avion&nbsp;?</blockquote>
		<section>
			 <ul>
					<li><a href="index.html">Accueil JS</a></li>
					<li><a href="avecstyle.html">Accueil CSS</a></li>
					<li><a href="htmlnu.html">Accueil HTML</a></li>
					<li><a href="quizz.html">Quizz</a></li>
      		</ul>
    		<h2 title="En bref">Le contrôle aérien dans les grandes lignes</h2>
    		<p>Les contrôleurs aériens, appelés parfois «&nbsp;aiguilleurs du ciel&nbsp;», travaillent soient dans un aéroport, soit dans un centre de contrôle en route. Ils sont en contact radio avec tous les avions pénétrant l'espace aérien dont ils ont la charge.</p>
    		<p>Les trois missions du contrôleur aérien sont&nbsp;:</p>
			<ol>
				<li><strong>Sécurité</strong> - bien entendu, la première mission du contrôleur aérien est d'assurer la sécurité des vols. Pour ce faire, il respecte des <em>normes de séparation</em>. Par exemple, lors d'un décollage ou d'un atterrissage, il ne doit pas y avoir plus d'un avion sur la piste&nbsp;; en l'air, les aéronefs sont séparés de 1000 pieds verticalement (300 mètres) ou 3 miles nautiques latéralement (environ 5km).</li>
				<li><strong>Environnement</strong> - les contrôleurs aériens assurent ensuite le respect de contraintes environnementales, comme les hauteurs minimales de survol des agglomérations pour limiter les nuisances sonores.</li>
				<li><strong>Optimisation</strong> - le but de cette dernière mission est d'optimiser les trajectoires afin de réduire les temps de vol.</li>
			</ol>\end{code2}
	
	\textbf{Deuxième h2}, puis h3 pour structurer les explications. Balises "figure", "img" et "figcation" pour insérer une image avec légende.
	
	\smallskip
	
	\begin{code2}
			<h2 title="En pratique">Concrètement, comment ça marche&nbsp;?</h2>

			<h3>Sectorisation de l'espace aérien</h3>
			<p>L'espace aérien est divisé en volumes, appelés <em>secteurs de contrôle</em>, qu'on peut se représenter comme des cubes pour simplifier. Chaque cube, et tous les avions qu'il contient, sont sous la responsabilité d'un binôme de contrôleurs. Lorsque le trafic aérien s'intensifie, ces cubes peuvent à leur tour être diviser en de plus petits cubes, afin de réduire la charge de travail des contrôleurs, ainsi que l'occupation de fréquence (voir Les outils du contrôleur aérien - la radio).</p>
			<p>Tout au long de son vol, un vol commercial est en contact radio avec les contrôleurs aériens, depuis la mise en route des moteurs jusqu'à leur extinction à l'arrivée. À chaque changement de secteur de contrôle (chaque fois qu'il passe d'un «&nbsp;cube&nbsp;» à un autre), il change de fréquence radio et contacte les contrôleurs du secteur suivant.</p>
			<figure>
				<img src="illustrations/coupeVerticale.png" alt="Schéma simplifié d'un vol Marseille-Nice">
				<figcaption>Coupe Verticale des secteurs de contrôle traversés par un vol Marseille-Nice</figcaption>
			</figure>
\end{code2}

	\textbf{Tableau} pour placer une image à côté d’un texte. Cette structure disparaîtra dans les pages stylées avec CSS.
	
	\smallskip 	
	
	\begin{code2}
			<ul>
				<li><strong>La radio</strong></li>
			</ul>
			<table>
				<tbody>
					<tr>
						<td><img src="illustrations/casque.png" alt="Une photo de casque"></td>
						<td>Les échanges entre pilotes et contrôleurs se font uniquement par radio, un équipement inventé à la fin du XIX<sup>ème</sup> siècle et qui a peu évolué depuis&nbsp;. En particulier, une seule station peut émettre à la fois ; dans le cas contraire les émissions sont brouillées. Cela arrive dans le cas d'une trop grande occupation de fréquence : la fréquence est saturée, pilotes et contrôleurs n'arrivent plus à passer leurs messages. Pilotes et contrôleurs communiquent dans un langage standardisé, appelé <em>phraséologie</em>, dont le but est d'améliorer la compréhension.</td>
					</tr>
				</tbody>
			</table>
				\end{code2}
	
	\textbf{Petit encart "zoom sur..."} pour les plus motivés, ou ceux qui veulent avoir tout juste au quizz. Les boutons ne fonctionnent pas par manque de temps, l’idée était de créer deux nouvelles pages html avec juste cet encart modifié, et des liens cliquables (<a href=...) vers ces pages. Les boutons ne fonctionnent que dans la page "Accueil JS". L’entête de l’encart est sous forme de tableau, pour la mise en forme. Puis bas de page, contenant deux liens externes en rapport avec le sujet.
		
		\smallskip
		
	\begin{code2}
		<!--si "section" ou "article" à la place de "div", validator.w3.org me reproche de ne pas mettre de titres ('warning : lack of heading')-->
		<div>
			<table>
				<thead>
					<tr>
						<td><strong>Zoom sur...</strong></td>
						<!--rendre les boutons cliquables-->
						<td><button type="button">&lt;&lt;</button></td>
						<td>un strip</td>
						<td><button type="button">&gt;&gt;</button></td>
					</tr>
				</thead>
				<tbody>
					<tr><td colspan="4"><img src="illustrations/strip.png" alt="La photo d'un strip"></td></tr>
					<tr><td colspan="4">Un strip contient un grand nombre d'informations, souvent codifiée. L'espace de gauche donne les éléments généraux du vol. Entre autres, TVF7217 (à prononcer «&nbsp;france soleil soixante-douze dix-sept&nbsp;») est l'<em>indicatif du vol</em>&nbsp;; «&nbsp;B738&nbsp;» est le <em>type d'avion</em>, un Boeing 737-800&nbsp;; «&nbsp;LFTH&nbsp;» et «&nbsp;LFRS&nbsp;» sont les <em>codes <abbr title="Organisation de l'Aviation Civile Internationale">OACI</abbr></em> des aéroports de départ, Toulon-Hyères, et d'arrivée, Nantes-Atlantique. Un peu plus à droite, «&nbsp;JULEE PADKO FJR&nbsp;» sont les noms des points que le pilote a prévu de survoler, <em>sa route</em>, en dessous desquels on peut également lire leurs heures de survol prévues. Encore à droite, les nombres «&nbsp;2000 ... 170&nbsp;» représentent les <em>altitudes</em> que le contrôleur est susceptible d'autoriser.</td></tr>
				</tbody>
			</table>
		</div>
		<hr>
		<div>
			<p>Ce site a été réalisé dans le cadre d'un TP, pour la première année de licence d'informatique de l'<abbr title="Institut d'Enseignement à Distance">IED</abbr> Paris 8.</p>
			<p>Vous voulez en savoir plus sur le contrôle aérien&nbsp;? Voici un <a href="https://devenir.controleuraerien.fr/">site</a> pas plus joli que le mien, sponsorisé par un syndicat.</p>
			<p>Un autre <a href="https://www.enac.fr/fr/mcta-controleur-aerien">lien</a> pour en apprendre plus sur la formation des contrôleurs, vers le site de l'<abbr title="École Nationale de l'Aviation Civile">ENAC</abbr>, seule école en France à former des contrôleurs aériens.</p>
		</div>
  	</body>\end{code2}
	
\medskip

\textbf{- Ajoutez une C.E (Copie d'Ecran) de votre éditeur de textes ainsi qu'une (ou plusieurs selon) C.E(s) de votre page affichée dans votre navigateur.}

\medskip

Voir fig~\ref{c2editeur} pour l’éditeur de texte, et fig~\ref{c2navigateur1}, fig~\ref{c2navigateur2}, fig~\ref{c2navigateur3}, fig~\ref{c2navigateur4} pour la page dans le navigateur. 

		
		\begin{figure}[h]
		\includegraphics[width=17cm, center]{CE/c2navigateur1.png}
		\caption{Chapitre 2 - CE de la page html sans style dans firefox (1/4)}
		\label{c2navigateur1}
		\end{figure}	

\textbf{- Ajoutez enfin une C.E de votre page qui est passée par un validateur et qui ne contient donc aucune erreur.} Voir fig~\ref{c2validateur}.

		\begin{figure}[h]
		\includegraphics[width=12cm, center]{CE/c2validateur.png}
		\caption{Chapitre 2 - CE de la validation de la page html sans style par validator.w3.org}
		\label{c2validateur}
		\end{figure}	

\textbf{- Joignez en archive attachée le contenu du Document Root (htdocs) portant le numéro de l’exercice}. Cf archive.

	\subsection*{Exercice 2.2 (Bonus)}
	
	Deux éléments à compléter pour un Bonus :

Ajoutez une courte vidéo (> 1min), en vous présentant en quelques mots (ou dans la thématique retenue) dans votre page au format approprié. Vous pouvez bien évidemment donner un côté humoristique ou ludique à celle-ci si le coeur vous en dit. (Joignez ensuite celle-ci à l’archive et faites-là appparaître dans une C.E dans le PDF)

Ajoutez des métadonnées au format LD-JSON en plus de celles de l’Exercice 2.1.

\textbf{Réponse :}

\section*{Chapitre 3 - Les Feuilles de Style en Cascade}
	\addcontentsline{toc}{section}{Chapitre 3 - Les Feuilles de Style en Cascade}

	\subsection*{Exercice 3.1 (À Rendre)}
	\addcontentsline{toc}{subsection}{Exercice 3.1 (À Rendre)}
	
	A L’aide d’un ou du même éditeur de Textes utilisé précédemment :

	Dupliquez la page web personnelle que vous avez construite au chapitre précédent et insérez dans cette dernière les CSSv3 en vous attachant à utiliser les balises float, clear, span et div. 
	
Comme vous disposez de deux pages distinctes sur votre serveur, ajoutez un menu ou une barre de navigation de votre choix à votre page d’accueil.

Gardez bien évidemment les métadonnées descriptives sur ces deux pages.

Attention : il est important qu’il ne s’agisse pas d’une nouvelle page sur un nouveau sujet mais d’une modification par simple copie de la page utilisée dans le chapitre précédent.

Dans le PDF, après avoir rappelé, comme à l’habitude, l’énoncé de l’exercice :

\textbf{- Indiquez l’URL de votre nouvelle page "stylée"}

\url{https://atco.000webhostapp.com/avecstyle.html}

\textbf{- Faites un C.C (Copier-Coller) de la feuille de Style dans son intégralité en ajoutant des commentaires sur son utilisation et vos choix.}
	
\medskip

Le style est entièrement contenu dans la feuille de style, à la manière du site CSSZengarden. Les couleurs du site sont inspirés des palettes proposés sur le site \href{https://static.vecteezy.com/system/resources/previews/000/641/263/original/website-color-palette-ideas-vector.jpg}{vecteezy.com}. Code CSS complet (le code html est en annexe) :

\begin{code2}
/* Chapitre 3 du cours "Utilisation d'Ordinateurs en Réseau" de P. Kislin 
Palette de couleurs : https://static.vecteezy.com/system/resources/previews/000/641/263/original/website-color-palette-ideas-vector.jpg*/

/* la citation d'intro, juste après le titre. Les lettres sont plus espacées pour un effet de légèreté */
.citation {
	text-align: right;
	margin:0;
	padding:10pt;
	font: italic normal lighter 1.2em Verdana, Arial, cursive;
	color: #5577D1;
	letter-spacing: +2px;
}

/* le titre n'a pas de style particulier, seul son isolement le rend spécial */
h1 {
  padding: 10pt;
  margin: 50pt;
  font: small-caps Georgia, Verdana, sans-serif;
}

body {
    font-family: 'source-code-pro', Arial, sans-serif;
    margin: 15%; /* la marge de 15% permet de s'adapter à toute taille d'écran */
    color: #2A3F74;
    text-align: justify;
    line-height: 1.5; /*l'espacement entre les lignes est plus grand pour plus de confort de lecture */
}

img {
	padding: 0pt 10pt; /* une mini marge latérale uniquement */
	max-width: 100%; /* pour ne pas dépasser de la page quand on rétrécit le navigateur */
}

/* barre de navigation - merci https://www.w3schools.com/Css/css_navbar_horizontal.asp */
.navbar {
  list-style-type: none; 
  margin: 0;
  padding: 0;
  overflow: hidden;
  background-color: #F6F3EC;
}
\end{code2}
\begin{code2}
/* chaque lien de la navbar apparait dans un rectangle */
li a {
  display: block;
  color: #2A3F74;
  text-align: center;
  padding: 14px 16px;
  text-decoration: none; /* enlève le soulignage */
}
	
/* quand la souris passe sur un rectangle, sa couleur devient plus foncée */
li a:hover:not(.active) {
  background-color: #EFE9DC;
}
 
/* signale la page sur laquelle on est dans la navbar */
.active {
  background-color: #2A3F74;
  color: white;
}
/* fin barre de navigation */

/* images intégrées dans un texte, à gauche ou à droite */
.r {
	float: right;
}

.l {
	float: left;
}

/* Permet de contenir une image dans la section à laquelle elle appartient 
(pas de dépassement) ; source w3schools */
.clearfix { 
	content: "";
	clear: both;
	display: table;
}

/* Certains mots sont mis en valeur */
.jargon {
	color: #5577D1;
	font-style: oblique;
}
\end{code2}
\begin{code2}

/* encart vers la fin */
.zoom {
	border-radius: 5%; /* bords un peu arrondi pour plus de douceur */
	background-color: #F6F3EC;
	padding: 10%; /* important, rend l'encadré plus aéré */
}

/* style de l'entête de l'encart "zoom sur" (liste sans point) */
.pointless {
	list-style-type: none; 
  overflow: hidden;
}

.inline {
	display: inline;
	padding: 10pt;
}

/* pour que la taille des images ne dépasse pas celle de la fenêtre, img centrée */
.pastropgrand {
	max-width: 100%;
	padding: 0;
	display: block;
	margin-left: auto;
	margin-right: auto;
}

.basdepage {
	text-align: center;
	color: #5577D1;
}

/* quizz (cf chapitre 5) */
.erreur {
	color: #E06C43;
	font-style: oblique;
}

.questions {
	background-color: #F6F3EC;
	padding: 14px 16px;
}
\end{code2}

\textbf{- Ajoutez une C.E (Copie d’Ecran) de votre page affichée avec le style choisi dans votre navigateur.}

Voir fig~\ref{c3navigateur1}, fig~\ref{c3navigateur2}, fig~\ref{c3navigateur3}, fig~\ref{c3navigateur4} pour la page stylée dans le navigateur.

		\begin{figure}[h]
		\includegraphics[width=14cm, center]{CE/c3navigateur1.png}
		\caption{Capture d’écran de la page html stylée dans firefox (1/4)}
		\label{c3navigateur1}
		\end{figure}	
		


\textbf{- Joignez en archive attachée le contenu du Document Root (htdocs) qui contiendra, en plus, la feuille de style retenue.} Cf archive.

\newpage

	\subsection*{Exercice 3.2 (Bonus)}
	
	Rendez votre page "responsive" en utilisant un Framework CSS de votre choix (Bootstrap,...). Si vous réalisez le bonus, ce rendu adaptatif se substituera à la première question de l’exercice 3.1.

Remarque : Il est tout à fait possible, en vous inspirant du site CSSZenGarden (cf lien), de produire deux feuilles de style différentes qui pourront s’appliquer à un même contenu.

Pensez à bien ajouter les copies d’écran de l’affichage sur deux terminaux différents (par exemple : sur ordinateur et sur smartphone) ou avec les deux feuilles de styles si vous choisissez cette alternative.

\textbf{Réponse :}

\newpage

\section*{Chapitre 4 - Le Protocole HTTP, Clients et Serveurs Web}
	\addcontentsline{toc}{section}{Chapitre 4 - Le Protocole HTTP, Clients et Serveurs Web}

	\subsection*{Exercice 4.1 (À Rendre)}
	\addcontentsline{toc}{subsection}{Exercice 4.1 (À Rendre)}
	
	En utilisant deux ordinateurs connectés sur le réseau..., à l’aide d’une VM, ou en local, sur la même machine, avec deux fenêtres de Terminal ouvertes :

Recopiez un gros fichier entre deux machines (ou entre deux terminaux) en utilisant netcat.

Pensez à définir ce que "gros fichier" signifie... A l’heure actuelle, il s’agira vraisemblablement d’un fichier > 1Go sauf si votre ordinateur est très très ancien ou votre connexion ADSL est très lente.

Mesurez le temps que ça prend et en déduire le débit de la connexion entre les deux machines (probablement en Go ou en Mo par seconde).

N’oubliez pas d’expliquer les commandes utilisées, de réaliser les C.E(s) d’illustration et de citer au besoin vos toutes sources. Pour obtenir des fichiers de tests, vous pouvez utiliser le site TestDebit.

\textbf{Réponse :}

Pour cet exercice, j’ai utilisé deux ordinateurs, l’un sous Linux et l’autre sous Windows, tous deux connectés au même réseau wifi. J’ai transféré une vidéo de 735Mo. Ce n’est pas si « gros » que ça, et des fichiers test de 1Go étaient proposés sur le site TestDebit, mais je voulais tester avec un vrai fichier pour pouvoir vérifier le bon fonctionnement du transfert. 735Mo, ça reste raisonnable quand même, me semble-t-il.

	\begin{figure}[h]
	\includegraphics[width=8cm, center]{netcat/SPpropertiesLinux.png}
	\end{figure}

\begin{enumerate}
	\item \textbf{À la recherche de mon adresse IP }
	
	\leftline{- Sur Linux, avec la commande {\sffamily hostname -I} }
	
	\leftline{(en fait, on s’en moque, cet ordinateur sera l’émetteur du fichier)}
	
	\begin{figure}[h]
	 \includegraphics[width=5cm, left]{netcat/adresseipLinux.png}
	\end{figure}		 
	
	\rightline{- Sur Windows, via les paramètres réseau }
	
	\begin{figure}[h]
	\includegraphics[width=8cm, right]{netcat/adresseipv4.png}
	\end{figure}	
	
	\item \textbf{Vérification de l’intégrité du fichier (comme conseillé sur le site it-connect.fr) et connexion des ordinateurs :}
	
		\leftline{- Sur Linux, l’émetteur, on note l’empreinte numérique du fichier}
		\leftline{avec la commande {\sffamily md5sum}, puis on envoie le fichier à l’adresse IP}
		\leftline{de l’autre ordinateur avec la commande }
		\leftline{{\sffamily netcat -q 0 192.168.1.23 8000 < [nom du fichier]}.}
		\leftline{L’option {\sffamily -q 0} ferme automatiquement la connexion une fois}
		\leftline{le fichier transféré (elle restait active indéfiniment, sinon).}
		\begin{figure}[h]
	 	\includegraphics[width=14cm, left]{netcat/md5_netcat.png}
		\end{figure}	
		
		\rightline{Connexion établie et fichier récupéré de l’autre côté avec la commande }
		\rightline{{\sffamily ncat -Cv -p 8000 > [adresse où copier le fichier]}}
		\rightline{Ensuite, vérification de l’intégrité du fichier avec la commande}
		\rightline{{\sffamily certUtil -hashfile [adresse du fichier]}}
		\rightline{Les empreintes correspondent, et le film fonctionne, hourra !}
		\begin{figure}[h]
		\includegraphics[width=16cm, right]{netcat/ncat_md5.png}
		\end{figure}	
		
\end{enumerate}		

		{\em Note : sur le site it-connect.fr, il est précisé que l’ordre des commandes est important, et qu’il faut ouvrir la session netcat côté demandeur en premier (ordinateur Windows dans mon cas). Quand je suivais cet ordre, le fichier était mal copié : les empreintes numériques ne correspondaient pas, et le film tournait mal.}
		
		La commande {\sffamily time} lancée en même temps que netcat nous informe du temps utilisé pour le transfert : 1 minute et 16 secondes. 
		
		\begin{figure}[!htb]
		\begin{minipage}{0.5\textwidth}
		\centering	  
	  \includegraphics[width=3.3cm]{netcat/testdebitLinux.png}
	  \caption{test de débit Linux (émetteur)}
		\end{minipage} \hfill
		\begin{minipage}{0.5\textwidth}
		\centering	 
		\includegraphics[width=3cm]{netcat/testdebitWin.png}
	  \caption{test de débit Windows (récepteur)}
		\end{minipage} 
	\end{figure}
	
	Des tests de débits sur le site testdebit.info nous indiquent que l’ordinateur émetteur envoie des données avec une moyenne de 102.6Mb/s, le récepteur en reçoit à une moyenne de 336.8Mb/s. La vitesse d’émission est le facteur limitant. $t = 735 * 8 / 102.6 = 57.3$ À cette vitesse, notre fichier de 735Mo aurait dû être reçu en 57 secondes, or il a mis 30 pour cent de temps en plus. Ce delta peut avoir plusieurs explications, la première nous est donné dans les tests de débit : la latence, c’est-à-dire le temps de transmission/réponse des requêtes. Une autre explication est donnée par le site filemail.com :
	
	\begin{quote}
	«~TCP n'utilise qu'une fraction de votre bande passante grâce à des éléments tels que le contrôle de la congestion et l'accusé de réception du paquet de données par le récepteur.~» \footnote{\url{https://www.filemail.com/fr/file-transfer-time-calculator}}
	\end{quote}

	\subsection*{Exercice 4.2 (À Rendre)}
	\addcontentsline{toc}{subsection}{Exercice 4.2 (À Rendre)}
	
	Sur votre ordinateur en local (sur lequel vous aurez installé XAMPP), en virtualisation (VirtualBox, VMWare, Docker,...) ou chez votre hébergeur gratuit (selon) :

Installez une Application Web de votre choix (CMS, LMS, Wiki,....) et compléter une page à minima avec des contenus relatifs à la thématique choisie (cf chapitre 2).

Expliquez et Illustrez chacune des phases de l’installation à l’aide de commentaires et de copies d’écran (C.E(s)).

Pour Vous Aider : Vous pouvez vous rendre sur le site Bitnami où vous trouverez une collection de modules XAMPP à installer.

\textbf{Réponse :}


	\subsection*{Exercice 4.3 (Bonus)}
	
	A réaliser en local sur votre machine, avec XAMPP, Python, NodeJS,... (selon) installé.s ou à l’aide d’une image docker (cf QR-codes pour aller plus loin) :

Installez (une application au choix) : Apache Nifi, ElasticSearch(*)(+ kibana), une base NoSQL (MongoDB, Redis, Neo4j,Cassandra, CouchDB, etc.), un forum de discussion (Discourse, Talkyard, NodeBB (si NodeJS), PhpBB, etc.), un serveur de
jeux, un serveur de Chat (ou IRC),un full-stack MERN ou MEAN, Streamlit (Python),... ou encore une seconde application web prise dans la liste de Bitnami en plus de celle de l’exercice 4.1.

N’oubliez pas d’ajouter un contenu minimal et d’illustrer votre PDF avec les C.E(s) des différentes phrases d’installation et du rendu final.

Remarque(*) : Au besoin, vérifiez que la version d’ElasticSearch utilisée est encore "Open Source" (licence Apache 2.0). Ce n’est peut-être plus le cas.

Un exemple de traitement de données avec Streamlit Le vélo à Paris, qui vous sera utile pour celles et ceux qui souhaiteraient continuer en Master Big Data à l’IED.

\textbf{Réponse :}

\section*{Chapitre 5 - Lire et traiter des Informations via un Page Web avec PHP}
	\addcontentsline{toc}{section}{Chapitre 5 - Lire et traiter des Informations via un Page Web avec PHP}

	\subsection*{Exercice 5.1 (À Rendre)}
	\addcontentsline{toc}{subsection}{Exercice 5.1 (À Rendre)}

Ajoutez à votre page personnelle un formulaire comportant à minima 5 champs différents, Traitez-le avec du PHP (ou avec le langage de votre choix) en utilisant obligatoirement une base de données relationnelle (MySQL ou équivalent).

Pour que ce soit intéressant, il faut que la page écrive des données dans la base mais aussi qu’elle les relise pour affichage. Un exemple simple est un livre d’or, une demande de contacts, un ’laissez votre avis’, ..., qui permettra de laisser données et informations sur la page et bien évidemment d’afficher celles-ci pour validation.

Pour vous aider, vous pouvez utiliser un outil d’administration comme PhpMyAdmin (ou équiv.). Pensez à rendre le plus sûr possible les saisies de l’utilisateur notammment sur les champs "date" et "adresse email" qui devront apparaître dans votre formulaire.

N’oubliez pas d’expliquer les requêtes utilisées (SQL ou équiv), les traitements opérés, de réaliser les C.E(s) d’installation et d’illustration du fonctionnement de votre forumulaire et de joindre toutes les sources à votre dossier Htdocs en archive attachée à votre envoi.

\textbf{Réponse :} Pour cet exercice, j’ai réalisé un quizz que vous trouverez à l’adresse \url{https://atco.000webhostapp.com/quizz.html}. Pour un aperçu de ce quizz dans firefox, voir fig~\ref{c5navigateur1}. Vous trouverez ici des extraits de code afin d’en comprendre le fonctionnement et la logique, pour le code complet, je vous invite à vous reporter aux annexes.

\textbf{Traitement des réponses avec PHP}

Les réponses à ce quizz sont traitées par PHP via la méthode POST, pour plus de sécurité qu’avec GET. Il comporte :
\begin{itemize}
	\item un champs 'nom' obligatoire dans lequel on ne peut entrer que certains caractères. En cas d’erreur de saisie ou de champs vide, une erreur s’affiche (cf fig~\ref{c5erreurs}) et les données rentrées ne sont pas enregistrées dans la base de donnée.
	\smallskip
	\begin{code3}
			if (empty($_POST['nom'])) {
				$nomErr = $obl;
			} else {
				$nom = nettoyer($_POST['nom']);
				// vérifie si nom contient seulement lettres, chiffres et espaces
				if (!preg_match("/^[0-9a-zA-Z- Ééèô]*$/",$nom)) {
  				$nomErr = "seulement lettres, chiffres et espaces autorisés";
  				}
			}\end{code3}
  	
  	\item un champs 'courriel', facultatif. Pour s’assurer que le format du courriel soit cohérent, on utilise la fonction native {\sffamily filter\_var}.
  	\smallskip
  	\begin{code3}
  		if (!empty($_POST['courriel'])) {
				$courriel = nettoyer($_POST['courriel']);
				if ((!filter_var($courriel, FILTER_VALIDATE_EMAIL)) || (!preg_match("'", $courriel))) {
  				$courrielErr = "adresse non valide";
				}
			}\end{code3}
  	
	\item un champs 'date' particulièrement inutile, si ce n’est pour s’amuser à rentrer des dates absurdes. Si le champs est laissé vide, la date retenue sera la date du jour. Pour s’assurer de la validité de la date, si elle est rentrée, on utilise la fonction native {\sffamily validateDate}. Autrement, une erreur s’affiche (cf fig~\ref{c5erreurs}).
	\smallskip
	\begin{code3}
				date_default_timezone_set("Europe/Paris");
  			if(!empty($_POST['date'])) {
  				$date = $_POST['date'];
  				if (!validateDate($date)) {
  				$dateErr = "date non valide";
  				}
  			} else {
  				$date = date('d/m/Y');
  			}\end{code3}
  			
  	Si ces trois variables sont remplies correctement (donc messages d’erreur vides), la variable booléenne {\sffamily \$questionnaire\_ok} renvoie TRUE.
  	\smallskip
  	\begin{code3}
if ($nomErr == "" && $courrielErr == "" && $dateErr == "") {
  	$questionnaire_ok = TRUE;
}\end{code3}
  	
  	\begin{figure}[h]
		\includegraphics[width=10cm, center]{CE/c5erreurs.png}
		\caption{Capture d’écran de messages d’erreur dans firefox}
		\label{c5erreurs}
		\end{figure}	
		
	\item cinq questions à choix multiple de type "radio" (une seule réponse possible). Le code PHP {\sffamily if (isset(\$var) \&\& \$var=="") echo "checked";} permet à la réponse choisie par l’utilisateur de rester cochée. Exemple de la première question :
	\smallskip
	\begin{code3}
<p><strong>Question 1 :</strong> la norme de séparation radar est-elle respectée entre le SWW264 et le DIFCB&nbsp;?</p>
<p><img class="pastropgrand" src="illustrations/sepradar1.png" alt="image d'une séparation radar"></p>
<p><input type="radio" name="question1" <?php if (isset($rep_user[0]) && $rep_user[0]=="oui") echo "checked";?> value="oui">oui
		<input type="radio" name="question1" <?php if (isset($rep_user[0]) && $rep_user[0]=="non") echo "checked";?> value="non">non</p>\end{code3}
		
Les réponses de l’utilisateur au QCM sont concervées dans la table {\sffamily rep\_user} :
\smallskip
\begin{code3}
$rep_user = array();
for ($i = 0; $i < 5; $i++) {
  if (!empty($_POST[$nom_questions[$i]])) {
  	array_push($rep_user, $_POST[$nom_questions[$i]]);
  } else {
  	array_push($rep_user, "");
 }
}\end{code3}

\end{itemize} 
 	
\textbf{Base de donnée} J’ai honte de ma base de donnée... Prise par le temps, j’ai fait au plus simple sachant les fonctionnalités que je voulais apporter. Ce n’est ni optimisé, ni évolutif (l’ajout d’une question oblige à modifier ma table en ajoutant une colonne). Idéalement, il aurait fallu créer trois tables, une avec les données de l’utilisateur, une autre avec ses réponses, une dernière avec les questions et les réponses justes. Ce que j’ai fait, c’est une table unique, contenant les nom et courriel de l’utilisateur, la date, les 5 réponses, et 5 autres colonnes valant "1" en cas de réponse juste, "0" sinon. Ces dernières colonnes me permettent de calculer facilement le pourcentage de bonne réponse pour chaque question. Les réponses justes ne sont pas conservées dans la base de donnée, mais dans une table dans le code PHP :
\smallskip
\begin{code3}
$rep_correctes = array("oui", "B737", "phraséologie", "5000", "oui");\end{code3}$

Bref... La base de donnée a été créée via PhpMyadmin (fig~\ref{c5bdd} pour un aperçu de la table). La table a été créée avec le code suivant :

\begin{figure}[h]
		\includegraphics[width=19cm, center]{CE/c5bdd.png}
		\caption{CE d’un extrait de la table "questionnaire" dans PhpMyadmin}
		\label{c5bdd}
		\end{figure}	
\smallskip
\begin{code3}
CREATE TABLE id21044620_atcodb.questionnaire (
   clef INT(4) AUTO_INCREMENT PRIMARY KEY,
	nom VARCHAR(26),
	courriel VARCHAR(26),
	date VARCHAR(10),
	question1 VARCHAR(3),
	question2 VARCHAR(4),
	question3 VARCHAR(30)
);\end{code3}

Puis j’ai ajouté deux colonnes pour les deux questions suivantes, et 5 autres pour les scores à chaque question, avec des instructions du type 

{\sffamily ALTER TABLE id21044620\_atcodb.questionnaire ADD res1 INT(1);}

La connection à la base de donnée se fait ainsi (uniquement si l’utilisateur a bien rempli les trois premiers champs) :
\smallskip
\begin{code3}
if ($questionnaire_ok) {
	$conn = mysqli_connect("localhost", "id21044620_atco", "icna11b@Nice", "id21044620_atcodb");
	if (!$conn) {
		die("Connection failed : " . mysqli_connect_error());
	}
}\end{code3}$

et l’insertion des réponses de l’utilisateur dans la table de cette manière :
\smallskip
\begin{code3}
$sql_insert = "INSERT INTO id21044620_atcodb.questionnaire (`nom`, `courriel`, `date`, `question1`, `question2`, `question3`, `question4`, `question5`, `res1`, `res2`, `res3`, `res4`, `res5`) VALUES ('$nom', '$courriel', '$date', '$rep_user[0]', '$rep_user[1]', '$rep_user[2]', '$rep_user[3]', '$rep_user[4]', '$resultats_user[0]', '$resultats_user[1]', '$resultats_user[2]', '$resultats_user[3]', '$resultats_user[4]')";

/* test insertion */
if (mysqli_query($conn, $sql_insert)) {
	echo "Vos réponses ont bien été enregistrées.";
} else {
	echo "Error : " . $sql_insert . "<br>" . mysqli_error($conn);
}\end{code3}

La table {\sffamily \$resultats\_user} conserve les scores de l’utilisateur pour chaque question. Il a été défini un peu plus tôt dans le code par :
\smallskip
\begin{code3}
for ($i = 0; $i < 5; $i++) {
  if ($rep_user[$i] == $rep_correctes[$i]) {
  	array_push($resultats_user, 1);
  } else {
  	array_push($resultats_user, 0);
  }
}\end{code3}$

Je récupère le nombre de participant pour mes mini statistiques :
\smallskip
\begin{code3}
$sql_nb_participants = "SELECT COUNT(`clef`) FROM id21044620_atcodb.questionnaire";
$nb_participants = mysqli_fetch_row(mysqli_query($conn, $sql_nb_participants))[0];\end{code3}

\textbf{Affichage des réponses et statistique par question}

Pour chaque question, la page affiche si la réponse est juste ou non, avec la réponse juste le cas échéant, et quelques explications. Ensuite, une requête SQL récupère la somme de la colonne des scores à la question, et php calcule le pourcentage de bonnes réponses à cette question. Pour la première question, cela donne (cf fig~\ref{c5reponse}) :
\smallskip
\begin{code3}
<?php
	// affichage réponse si nom, courriel et date remplis correctement
	if ($questionnaire_ok) {
		// si bonne réponse
		if ($resultats_user[0]==1) {
			echo "<p class='jargon'>Bonne réponse ! ";
		} else {
			echo "<p class='erreur'>Perdu. ";
		}
		echo("Les deux avions sont à 2000ft (deuxième ligne de l'étiquette), mais ils sont séparés latéralement de 6,86NM. La norme est donc respectée.</p>");

		// affichage pourcentage bonne réponse à la question
		$sql_nb_res1_correct = "SELECT SUM(`res1`) FROM id21044620_atcodb.questionnaire";
		$nb_res1_correct = mysqli_fetch_row(mysqli_query($conn, $sql_nb_res1_correct))[0];
		echo "<p>" . round($nb_res1_correct/$nb_participants*100) . "&#37; des participants ont répondu correctement à cette question.</p>";
	}
?>\end{code3}

\begin{figure}[h]
		\includegraphics[width=12cm, center]{CE/c5reponse.png}
		\caption{Capture d’écran d’une réponse juste dans firefox}
		\label{c5reponse}
		\end{figure}	
		
\textbf{Quelques bugs...}

\begin{itemize}
	\item Si on actualise la page "reception\_questionnaire.php" après avoir soumis ses réponses et avoir vu les résultats, les réponses sont à nouveau enregistrées dans la base de donnée.
	
	\item À l’origine, dans l’expression rationnelle permettant de valider le nom de l’utilisateur, j’avais autorisé le caractère «~'~». Ça faisait apparaître une erreur liée à la base de donnée, puisque les entrées de la base de données sont entourées d’apostrophes simples. Problème corrigé.
	
	\item La fonction {\sffamily filter\_var} ne filtre pas non plus l’apostrophe simple, j’ai dû le rajouter à la main avec une fonction {\sffamily preg\_match}.
	
	\item Dans une version antérieure, j’affichais les noms de tous les participants au quizz. Un proche qui travaille dans l’informatique s’est frotté les mains à l’idée de spammer ma page. De manière générale, dépêchez-vous d’aller sur le site, deux "amis" se sont donnés pour mission de trouver toutes les failles et le faire planter :(
\end{itemize}

	\subsection*{Exercice 5.2 (Bonus)}

Réalisez le même formulaire, traité avec le langage de votre choix, et cette fois à l’aide d’une base NoSQL.

N’oubliez pas d’expliquer et d’illustrer comme dans l’exercice 5.1.

\textbf{Réponse :}

\section*{Chapitre 6 - Le Langage JavaScript}
	\addcontentsline{toc}{section}{Chapitre 6 - Le Langage JavaScript}

	\subsection*{Exercice 6.1 (À Rendre)}
	\addcontentsline{toc}{subsection}{Exercice 6.1 (À Rendre)}

Réalisez une copie de votre page initiale (comporant les CSS) et ajoutez à celle-ci du JavaScript afin d’obtenir, en utilisant le DOM, une page qui réagit aux actions de l’utilisateur.

Comme vous disposez maintenant de quatre pages distinctes sur votre serveur (en local ou chez votre hébergeur), ajoutez cette nouvelle page à votre menu ou à votre barre de navigation.
 
N’oubliez pas d’expliquer les méthodes DOM/JS utilisées, de C.C le code Javascript dans son intégralité, de réaliser une (ou plusieurs selon) C.E(s) de votre page JS réactive affichée dans votre navigateur et d’ajouter votre ou vos sources à votre dossier Htdocs en archive attachée à votre envoi.

\smallskip

\textbf{Réponse :} J’ai utilisé JavaScript pour modifier l’encadré "Zoom sur..." à la fin, et rendre les boutons réactifs. La page concernée est la page d’accueil du site, soit \url{atco.000webhostapp.com/index.html}. Le code principal JS est dans la balise "head" de la page html, il semblerait que ce soit l’usage. Six variables, ou plutôt constantes, ont été introduites, pour les trois paragraphes et trois titres différents.

\smallskip

\begin{code2}
const strip_txt = '<img class="pastropgrand" src="illustrations/strip.png" alt="La photo d'un strip">Un strip contient un grand nombre d'informations, souvent codifiée. L'espace de gauche donne les éléments généraux du vol. Entre autres, <strong>TVF7217</strong> (à prononcer «&nbsp;france soleil soixante-douze dix-sept&nbsp;») est l'<span class="jargon">indicatif du vol</span>&nbsp;; <strong>B738</strong> est le <span class="jargon">type d'avion</span>, un Boeing 737-800&nbsp;; <strong>LFTH</strong> et <strong>LFRS</strong> sont les <span class="jargon">codes <abbr title="Organisation de l'Aviation Civile Internationale">OACI</abbr></span> des aéroports de départ, Toulon-Hyères, et d'arrivée, Nantes-Atlantique. Un peu plus à droite, <strong>JULEE PADKO FJR</strong> sont les noms des points que le pilote a prévu de survoler (<span class="jargon">sa route</span>) en dessous desquels on peut également lire leurs heures de survol prévues. Encore à droite, les nombres <strong>2000 ... 170</strong> représentent les <span class="jargon">altitudes</span> ou <span class="jargon">niveaux de vol</span> que le contrôleur est susceptible d'autoriser.';

const strip_ttl = '<larger> un strip </larger>';\end{code2}

Deux fonctions "previous" et "next" permettent d’activer les boutons pour afficher un autre contenu pour l’encadré. Les fonctions dépendent de ce qui est déjà affiché. Elles font un usage intensif de la méthode "document.getElementById().innerHTML vue dans le cours.

\smallskip

\begin{code2}
			function previous() {
				if (document.getElementById("par_zoom").innerHTML == strip_txt) {
					document.getElementById("par_zoom").innerHTML = phraseo_txt;
					document.getElementById("titre_zoom").innerHTML = phraseo_ttl;
				} else if (document.getElementById("par_zoom").innerHTML == sepradar_txt) {
					document.getElementById("par_zoom").innerHTML = strip_txt;
					document.getElementById("titre_zoom").innerHTML = strip_ttl;
				} else {
					document.getElementById("par_zoom").innerHTML = sepradar_txt;
					document.getElementById("titre_zoom").innerHTML = sepradar_ttl;
				}
			}\end{code2}

Le code HTML de l’encadré devient, avec une dernière utilisation de "document.getElementById().innerHTML :

\smallskip

\begin{code2}
		<div class="clearfix zoom">
			<ul class="pointless">
				<li class="l inline"><strong>Zoom sur...</strong></li>
				<li class="inline r"><button type="button" onclick="next()"> &gt;&gt;</button></li>
				<li id="titre_zoom" class="inline r"> un strip </li>
				<li class="inline r"><button type="button" onclick="previous()"> &lt;&lt;</button></li>
			</ul>
			<p id="par_zoom"></p>
		</div>
		<script>
				document.getElementById("par_zoom").innerHTML = strip_txt;
		</script>\end{code2}
		
Le reste est inchangé ; le code complet est en annexe. Exemple d’affichage dans un navigateur : voir fig~\ref{c6navigateur3}, fig~\ref{c6navigateur2} et fig~\ref{c6navigateur1}.


		
		\begin{figure}[h]
		\includegraphics[width=18cm, center]{CE/c6navigateur3.png}
		\caption{Capture d’écran de l’encadré "Zoom sur" dans firefox (1/3)}
		\label{c6navigateur3}
		\end{figure}


	\subsection*{Exercice 6.2 (Bonus)}
	
Dans la page actuelle ou dans une nouvelle page, insérez des données au Format JSON prises dans l’un des jeux de données disponibles dans l’Open Data.

L’affichage des données (idéalement filtrées) sera formatée sous la forme d’un tableau, d’un graphique ou d’une carte.

(Voir par exemple : https://www.stat4decision.com/fr/10-sites-de-reference-open-data/

https://data.enseignementsup-recherche.gouv.fr/pages/home/

https://www.data.gouv.fr/fr/

Celles et ceux qui auraient installé Streamlit peuvent utiliser ce framework pour afficher les données avec Python.

\textbf{Réponse :}


\section*{Chapitre 7 - Internet en dehors du Web}
	\addcontentsline{toc}{section}{Chapitre 7 - Internet en dehors du Web}

	\subsection*{Exercice 7.1 (À Rendre)}
	\addcontentsline{toc}{subsection}{Exercice 7.1 (À Rendre)}
	
(1) Après les avoir compressés en une seule archive, transférez les sources(*) de votre site (répertoire "htdocs" ou Document Root) de votre hébergeur (ou en local dans un autre dossier selon) en utilisant le protocole FTP.

(2) En utilisant les commandes netcat, ssh, scp, wget, git,... (selon), déposez votre archive (obtenue en (1)) dans un service d’hébergement gratuit de fichiers de votre choix ou sur un serveur de gestion de versions décentralisé (Github, Gitea, GitLab, BitBucket, SVN, Mercurial, SourceForge, ...)

Comme à l’habituée, n’oubliez pas de commenter les différentes opérations et d’illustrer celles-ci avec des C.E(s). Insérez également dans votre réponse le ou les liens correspondant.s afin que je puisse accéder à distance à votre archive.

Sources (*) : N’oubliez pas d’ajouter à votre archive, les sources du chatbot de l’exercice 7.2.

\textbf{Réponse :}


	\subsection*{Exercice 7.2 (Mini-Projet - À Rendre)}
	\addcontentsline{toc}{subsection}{Exercice 7.2 (Mini-Projet - À Rendre)}
	
Avec le langage et/ou le framework de votre choix, réalisez un mini-chatbot qui puisse dialoguer avec un utilisateur humain.

Ce chatbot doit pouvoir répondre à des "bonjour" ..., "bonsoir" ..., et à un minimum de 5 questions simples de votre choix.

Si vous le souhaitez, vous pouvez développer votre chatbot en local, chez votre hébergeur, ou à l’aide d’assistants comme IBM
Watson, Google DialogFLow, Amazon Lex (AWS) ou Alexa, Chatfuel,...

Insérez dans votre réponse les codes commentés (C.C), et bien évidemment les C.E(s) des (>= 5) exemples d’utilisation.

\textbf{Réponse :}

\newpage

\appendix
\section{Annexes au chapitre 2}

Code complet de la page sans style ni interactivité htmlpur.html :

\smallskip

\begin{code2}
<!DOCTYPE html>
<html lang="fr">
  	<head>
    	<title>Contrôleur aérien</title>
    	<meta charset="utf-8">
    	<meta name="description" content="Une première approche(haha) du métier de contrôleur aérien. Réalisé dans le cadre du cours 'Utilisation d'Ordinateurs en Réseau' de P. Kislin">
    	<meta name="url" content="atco.000webhostapp.com/htmlnu.html">
    	<meta name="author" content="OD">
    	<meta name="keywords" content="contrôle aérien, contrôleur aérien, aiguilleurs du ciel, ICNA, ATCO, navigation aérienne, DGAC">
  	</head>

<!--barre de navigation-->
  	<body>
    	<h1 title="Un peu plus près des étoiles">Métier : contrôleur aérien</h1>
    	<blockquote>Le contrôleur aérien, c'est celui qui contrôle les billets dans l'avion&nbsp;?</blockquote>
		<section>
			 <ul>
					<li><a href="index.html">Accueil JS</a></li>
					<li><a href="avecstyle.html">Accueil CSS</a></li>
					<li><a href="htmlnu.html">Accueil HTML</a></li>
					<li><a href="quizz.html">Quizz</a></li>
      		</ul>
<!--corps du site-->      		
    		<h2 title="En bref">Le contrôle aérien dans les grandes lignes</h2>
    		<p>Les contrôleurs aériens, appelés parfois «&nbsp;aiguilleurs du ciel&nbsp;», travaillent soient dans un aéroport, soit dans un centre de contrôle en route. Ils sont en contact radio avec tous les avions pénétrant l'espace aérien dont ils ont la charge.</p>
    		<p>Les trois missions du contrôleur aérien sont&nbsp;:</p>
			<ol>
				<li><strong>Sécurité</strong> - bien entendu, la première mission du contrôleur aérien est d'assurer la sécurité des vols. Pour ce faire, il respecte des <em>normes de séparation</em>. Par exemple, lors d'un décollage ou d'un atterrissage, il ne doit pas y avoir plus d'un avion sur la piste&nbsp;; en l'air, les aéronefs sont séparés de 1000 pieds verticalement (300 mètres) ou 3 miles nautiques latéralement (environ 5km).</li>
				<li><strong>Environnement</strong> - les contrôleurs aériens assurent ensuite le respect de contraintes environnementales, comme les hauteurs minimales de survol des agglomérations pour limiter les nuisances sonores.</li>
				<li><strong>Optimisation</strong> - le but de cette dernière mission est d'optimiser les trajectoires afin de réduire les temps de vol.</li>
			</ol>
\end{code2}
\begin{code2}
			<h2 title="En pratique">Concrètement, comment ça marche&nbsp;?</h2>

			<h3>Sectorisation de l'espace aérien</h3>
			<p>L'espace aérien est divisé en volumes, appelés <em>secteurs de contrôle</em>, qu'on peut se représenter comme des cubes pour simplifier. Chaque cube, et tous les avions qu'il contient, sont sous la responsabilité d'un binôme de contrôleurs. Lorsque le trafic aérien s'intensifie, ces cubes peuvent à leur tour être diviser en de plus petits cubes, afin de réduire la charge de travail des contrôleurs, ainsi que l'occupation de fréquence (voir Les outils du contrôleur aérien - la radio).</p>
			<p>Tout au long de son vol, un vol commercial est en contact radio avec les contrôleurs aériens, depuis la mise en route des moteurs jusqu'à leur extinction à l'arrivée. À chaque changement de secteur de contrôle (chaque fois qu'il passe d'un «&nbsp;cube&nbsp;» à un autre), il change de fréquence radio et contacte les contrôleurs du secteur suivant.</p>
			<figure>
				<img src="illustrations/coupeVerticale.png" alt="Schéma simplifié d'un vol Marseille-Nice">
				<figcaption>Coupe Verticale des secteurs de contrôle traversés par un vol Marseille-Nice</figcaption>
			</figure>

			<h3>Les outils du contrôleur aérien</h3>
			<figure>
				<img src="illustrations/positionControle2.png" alt="Une position de contrôle">
				<figcaption>Une position de contrôle dans la salle d'approche de Nice</figcaption>
			</figure>
			<p>
				Un binôme de contrôleurs aériens travaille sur une «&nbsp;position de contrôle&nbsp;», une console regroupant tous les outils nécessaires : platine radio, écrans radars, tableau de strip, platine téléphone et divers écrans d'information aéronautique. Les éléments essentiels sont :
			</p>
			<ul>
				<li><strong>La radio</strong></li>
			</ul>
			<table>
				<tbody>
					<tr>
						<td><img src="illustrations/casque.png" alt="Une photo de casque"></td>
						<td>Les échanges entre pilotes et contrôleurs se font uniquement par radio, un équipement inventé à la fin du XIX<sup>ème</sup> siècle et qui a peu évolué depuis&nbsp;. En particulier, une seule station peut émettre à la fois ; dans le cas contraire les émissions sont brouillées. Cela arrive dans le cas d'une trop grande occupation de fréquence : la fréquence est saturée, pilotes et contrôleurs n'arrivent plus à passer leurs messages. Pilotes et contrôleurs communiquent dans un langage standardisé, appelé <em>phraséologie</em>, dont le but est d'améliorer la compréhension.</td>
					</tr>
				</tbody>
			</table>	\end{code2}
\begin{code2}
			<ul>
				<li><strong>L'image radar</strong></li>
			</ul>
			<table>
				<tbody>
					<tr>
						<td>L'image radar sert pour la surveillance et la séparation des aéronefs. Elle permet au contrôleur de se représenter le trafic aérien dans l'espace, et de maintenir la norme de séparation radar (3Nm/1000ft) entre tous les aéronefs.</td>
						<td>
							<img src="illustrations/radar2.png" alt="Une image radar" height="230">
						</td>
					</tr>
				</tbody>
			</table>
			<ul>
				<li><strong>Les «&nbsp;strips&nbsp;»</strong></li>
			</ul>
			<table>
				<tbody>
					<tr>
						<td>
							<img src="illustrations/tableauStrip.png" alt="Une photo d'un tableau de strip" height="200">
						</td>
						<td>Un «&nbsp;strips&nbsp;» est une bandelette de papier sur laquelle sont inscrites toutes les informations utiles sur un vol : indicatif du vol (le nom utilisé pour l'appeler), aéroports de départ et d'arrivée, trajectoire prévu, etc. À chaque instruction qu'il donne à un avion, le contrôleur met à jour le strip, par exemple en entourant l'altitude autorisée.</td>
					</tr>
				</tbody>
			</table>
		</section>
<!--encart "zoom sur"-->
		<!--si "section" ou "article" à la place de "div", validator.w3.org me reproche de ne pas mettre de titres ('warning : lack of heading')-->
		<div>
			<table>
				<thead>
					<tr>
						<td><strong>Zoom sur...</strong></td>
						<!--rendre les boutons cliquables-->
						<td><button type="button">&lt;&lt;</button></td>
						<td>un strip</td>
						<td><button type="button">&gt;&gt;</button></td>
					</tr>
				</thead>\end{code2}
				\begin{code2}
				<tbody>
					<tr><td colspan="4"><img src="illustrations/strip.png" alt="La photo d'un strip"></td></tr>
					<tr><td colspan="4">Un strip contient un grand nombre d'informations, souvent codifiée. L'espace de gauche donne les éléments généraux du vol. Entre autres, TVF7217 (à prononcer «&nbsp;france soleil soixante-douze dix-sept&nbsp;») est l'<em>indicatif du vol</em>&nbsp;; «&nbsp;B738&nbsp;» est le <em>type d'avion</em>, un Boeing 737-800&nbsp;; «&nbsp;LFTH&nbsp;» et «&nbsp;LFRS&nbsp;» sont les <em>codes <abbr title="Organisation de l'Aviation Civile Internationale">OACI</abbr></em> des aéroports de départ, Toulon-Hyères, et d'arrivée, Nantes-Atlantique. Un peu plus à droite, «&nbsp;JULEE PADKO FJR&nbsp;» sont les noms des points que le pilote a prévu de survoler, <em>sa route</em>, en dessous desquels on peut également lire leurs heures de survol prévues. Encore à droite, les nombres «&nbsp;2000 ... 170&nbsp;» représentent les <em>altitudes</em> que le contrôleur est susceptible d'autoriser.</td></tr>
				</tbody>
			</table>
		</div>
<!--bas de page-->
		<hr>
		<div>
			<p>Ce site a été réalisé dans le cadre d'un TP, pour la première année de licence d'informatique de l'<abbr title="Institut d'Enseignement à Distance">IED</abbr> Paris 8.</p>
			<p>Vous voulez en savoir plus sur le contrôle aérien&nbsp;? Voici un <a href="https://devenir.controleuraerien.fr/">site</a> pas plus joli que le mien, sponsorisé par un syndicat.</p>
			<p>Un autre <a href="https://www.enac.fr/fr/mcta-controleur-aerien">lien</a> pour en apprendre plus sur la formation des contrôleurs, vers le site de l'<abbr title="École Nationale de l'Aviation Civile">ENAC</abbr>, seule école en France à former des contrôleurs aériens.</p>
		</div>
  	</body>
</html>
\end{code2}


		\begin{figure}[h]
		\includegraphics[width=18cm, center]{CE/c2editeur.png}
		\caption{Chap 2 - CE du début de la page html sans style dans sublime}
		\label{c2editeur}
		\end{figure}	
		
		\begin{figure}[h]
		\includegraphics[width=17cm, center]{CE/c2navigateur2.png}
		\caption{Chap 2 - CE de la page html sans style dans firefox (2/4)}
		\label{c2navigateur2}
		\end{figure}	
		
		\begin{figure}[h]
		\includegraphics[width=14cm, center]{CE/c2navigateur3.png}
		\caption{Chap 2 - CE de la page html sans style dans firefox (3/4)}
		\label{c2navigateur3}
		\end{figure}	
		
		\begin{figure}[h]
		\includegraphics[width=16cm, center]{CE/c2navigateur4.png}
		\caption{Chap 2 - CE de la page html sans style dans firefox (4/4)}
		\label{c2navigateur4}
		\end{figure}	
		
\section{Annexes au chapitre 3}

Code HTML complet :

\smallskip

\begin{code2}
<!DOCTYPE html>
<html lang="fr">
  	<head>
    	<title>Contrôleur aérien</title>
    	<meta charset="utf-8">
    	<link rel="stylesheet" type="text/css" href="style.css">
    	<meta name="description" content="Une première approche (haha) du métier de contrôleur aérien. Réalisé dans le cadre du cours 'Utilisation d'Ordinateurs en Réseau' de P. Kislin">
    	<meta name="url" content="atco.000webhostapp.com/avecstyle.html">
    	<meta name="author" content="OD">
    	<meta name="keywords" content="contrôle aérien, contrôleur aérien, aiguilleurs du ciel, ICNA, ATCO, navigation aérienne, DGAC">
  	</head>
  	<body>
    	<h1 title="Un peu plus près des étoiles">Métier : contrôleur aérien</h1>
    	<blockquote class="citation">Le contrôleur aérien, c'est celui qui contrôle les billets dans l'avion&nbsp;?</blockquote>
		<section>
			<ul class="navbar">
				<li class="l"><a href="index.html">Accueil JS</a></li>
				<li class="l"><a class="active" href="avecstyle.html">Accueil CSS</a></li>
				<li class="l"><a href="htmlnu.html">Accueil HTML</a></li>
				<li class="r"><a href="quizz.html">Quizz</a></li>
			</ul>\end{code2}
				\begin{code2}
    		<h2 title="En bref">Le contrôle aérien dans les grandes lignes</h2>
    		<p>Les contrôleurs aériens, appelés parfois «&nbsp;aiguilleurs du ciel&nbsp;», travaillent soit dans un aéroport, soit dans un centre de contrôle en route. Ils sont en contact radio avec tous les avions pénétrant l'espace aérien dont ils ont la charge.</p>
    		<p>Les trois missions du contrôleur aérien sont&nbsp;:</p>
			<ol>
				<li><strong>Sécurité</strong> - bien entendu, la première mission du contrôleur aérien est d'assurer la sécurité des vols. Pour ce faire, il respecte des <span class="jargon">normes de séparation</span>. Par exemple, lors d'un décollage ou d'un atterrissage, il ne doit pas y avoir plus d'un avion sur la piste&nbsp;; en l'air, les aéronefs sont séparés de 1000 pieds verticalement (300 mètres) ou 3 miles nautiques latéralement (environ 5km).</li>
				<li><strong>Environnement</strong> - les contrôleurs aériens assurent ensuite le respect de contraintes environnementales, comme les hauteurs minimales de survol des agglomérations pour limiter les nuisances sonores.</li>
				<li><strong>Optimisation</strong> - le but de cette dernière mission est d'optimiser les trajectoires afin de réduire les temps de vol.</li>
			</ol>
			
			<h2 title="En pratique">Concrètement, comment ça marche&nbsp;?</h2>

			<h3>Sectorisation de l'espace aérien</h3>
			<p>L'espace aérien est divisé en volumes, appelés <span class="jargon">secteurs de contrôle</span>, qu'on peut se représenter comme des cubes pour simplifier. Chaque cube, et tous les avions qu'il contient, sont sous la responsabilité d'un binôme de contrôleurs. Lorsque le trafic aérien s'intensifie, ces cubes peuvent à leur tour être diviser en de plus petits cubes, afin de réduire la charge de travail des contrôleurs, ainsi que l'occupation de fréquence (voir Les outils du contrôleur aérien - la radio).</p>
			<p>Tout au long de son vol, un vol commercial est en contact radio avec les contrôleurs aériens, depuis la mise en route des moteurs jusqu'à leur extinction à l'arrivée. À chaque changement de secteur de contrôle (chaque fois qu'il passe d'un «&nbsp;cube&nbsp;» à un autre), il change de fréquence radio et contacte les contrôleurs du secteur suivant.</p>
			<figure>
				<img class="pastropgrand" src="illustrations/coupeVerticale.png" alt="Schéma simplifié d'un vol Marseille-Nice">
				<figcaption>Coupe Verticale des secteurs de contrôle traversés par un vol Marseille-Nice</figcaption>
			</figure>

			<h3>Les outils du contrôleur aérien</h3>
			<figure>
				<img class="pastropgrand" src="illustrations/positionControle2.png" alt="Une position de contrôle">
				<figcaption>Une position de contrôle dans la salle d'approche de Nice</figcaption>
			</figure>\end{code2}
				\begin{code2}
			<p>Un binôme de contrôleurs aériens travaille sur une <span class="jargon"> position de contrôle</span>, une console regroupant tous les outils nécessaires : platine radio, écrans radars, <span class="jargon">tableau de strip</span>, platine téléphone et divers écrans d'information aéronautique. Les éléments essentiels sont :</p>
			<ul>
				<li><strong>La radio</strong></li>
			</ul>
			<p class="clearfix"><img class="l" src="illustrations/casque.png" alt="Une photo de casque"> Les échanges entre pilotes et contrôleurs se font uniquement par radio, un équipement inventé à la fin du XIX<sup>ème</sup> siècle et qui a peu évolué depuis&nbsp;. En particulier, une seule station peut émettre à la fois ; dans le cas contraire les émissions sont brouillées. Cela arrive dans le cas d'une trop grande occupation de fréquence : la fréquence est saturée, pilotes et contrôleurs n'arrivent plus à passer leurs messages. Pilotes et contrôleurs communiquent dans un langage standardisé, appelé <span class="jargon"> phraséologie</span>, dont le but est d'améliorer la compréhension. </p>
			<ul>
				<li><strong>L'image radar</strong></li>
			</ul>
			<p class="clearfix"><img class="r" src="illustrations/radar2.png" alt="Une image radar" height="230">L'image radar sert pour la surveillance et la séparation des aéronefs. Elle permet au contrôleur de se représenter le trafic aérien dans l'espace, et de maintenir la norme de séparation radar (3Nm/1000ft) entre tous les aéronefs.</p>
			<ul>
				<li><strong>Les «&nbsp;strips&nbsp;»</strong></li>
			</ul>
			<p class="clearfix"><img class="l" src="illustrations/tableauStrip.png" alt="Une photo d'un tableau de strip" height="200"> Un <span class="jargon"> strip</span> est une bandelette de papier sur laquelle sont inscrites toutes les informations utiles sur un vol : indicatif du vol (le nom utilisé pour l'appeler), aéroports de départ et d'arrivée, trajectoire prévu, etc. À chaque instruction qu'il donne à un avion, le contrôleur met à jour le strip, par exemple en entourant l'altitude autorisée.</p>
		</section>

<!--encart "zoom sur"-->
		<div class="clearfix zoom">
			<ul class="pointless"><!--rendre les boutons cliquables-->
				<li class="l inline"><strong>Zoom sur...</strong></li>
				<li class="inline r"><button type="button"> &gt;&gt;</button></li>
				<li class="inline r"> un strip </li>
				<li class="inline r"><button type="button"> &lt;&lt;</button></li>
			</ul>\end{code2}
				\begin{code2}
			<p><img class="pastropgrand" src="illustrations/strip.png" alt="La photo d'un strip">Un strip contient un grand nombre d'informations, souvent codifiée. L'espace de gauche donne les éléments généraux du vol. Entre autres, <strong>TVF7217</strong> (à prononcer «&nbsp;france soleil soixante-douze dix-sept&nbsp;») est l'<em>indicatif du vol</em>&nbsp;; <strong>B738</strong> est le <em>type d'avion</em>, un Boeing 737-800&nbsp;; <strong>LFTH</strong> et <strong>LFRS</strong> sont les <em>codes <abbr title="Organisation de l'Aviation Civile Internationale">OACI</abbr></em> des aéroports de départ, Toulon-Hyères, et d'arrivée, Nantes-Atlantique. Un peu plus à droite, <strong>JULEE PADKO FJR</strong> sont les noms des points que le pilote a prévu de survoler, <em>sa route</em>, en dessous desquels on peut également lire leurs heures de survol prévues. Encore à droite, les nombres <strong>2000 ... 170</strong> représentent les <em>altitudes</em> que le contrôleur est susceptible d'autoriser.</p>
		</div>
		<hr>
		
<!--bas de page-->
		<div class="basdepage">
			<p>Ce site a été réalisé dans le cadre d'un TP, pour la première année de licence d'informatique de l'<abbr title="Institut d'Enseignement à Distance">IED</abbr> Paris 8.</p>
			<p>Vous voulez en savoir plus sur le contrôle aérien&nbsp;? Voici un <a href="https://devenir.controleuraerien.fr/">site</a> pas plus joli que le mien, sponsorisé par un syndicat.</p>
			<p>Un autre <a href="https://www.enac.fr/fr/mcta-controleur-aerien">lien</a> pour en apprendre plus sur la formation des contrôleurs, vers le site de l'<abbr title="École Nationale de l'Aviation Civile">ENAC</abbr>, seule école en France à former des contrôleurs aériens.</p>
		</div>
  	</body>
</html>
\end{code2}

		\begin{figure}[h]
		\includegraphics[width=17cm, center]{CE/c3navigateur2.png}
		\caption{Chap 3 - CE de la page html stylée dans firefox (2/4)}
		\label{c3navigateur2}
		\end{figure}	
		
		\begin{figure}[h]
		\includegraphics[width=17cm, center]{CE/c3navigateur3.png}
		\caption{Chap 3 - CE de la page html stylée dans firefox (3/4)}
		\label{c3navigateur3}
		\end{figure}	
		
		\begin{figure}[h]
		\includegraphics[width=17cm, center]{CE/c3navigateur4.png}
		\caption{Chap 3 - CE de la page html stylée dans firefox (4/4)}
		\label{c3navigateur4}
		\end{figure}	

\section{Annexes au chapitre 5}

Code complet de la page du formulaire quizz.html :

\begin{code2}
<!DOCTYPE html>
<html lang="fr">
  	<head>
    	<title>Contrôleur aérien</title>
    	<meta charset="utf-8">
    	<link rel="stylesheet" type="text/css" href="style.css">
    	<meta name="description" content="Un quizz pour tester sa compréhension du contrôle aérien et de ses outils. Réalisé dans le cadre du cours 'Utilisation d'Ordinateurs en Réseau' de P. Kislin">
    	<meta name="url" content="atco.000webhostapp.com/quizz.html">
    	<meta name="author" content="OD">
    	<meta name="keywords" content="contrôle aérien, contrôleur aérien, aiguilleurs du ciel, ICNA, ATCO, navigation aérienne, DGAC">
  	</head>
  	<body>
  		<h1>L'heure du quizz</h1>
  		<p><blockquote class="citation">Serez-vous à la hauteur&nbsp;?</blockquote></p>

<!--La barre de navigation-->
  		<ul class="navbar">
			<li class="l"><a href="index.html">Accueil JS</a></li>
			<li class="l"><a href="avecstyle.html">Accueil CSS</a></li>
			<li class="l"><a href="htmlnu.html">Accueil HTML</a></li>
			<li class="r"><a class="active"  href="quizz.html">Quizz</a></li>
		</ul>\end{code2}
				\begin{code2}
<!--Le Questionnaire-->
		<section>
			<p class="erreur">* champs obligatoire</p>
			<!--methode POST pour une question de sécurité-->
			<form method="post" action="https://atco.000webhostapp.com/reception_quizz.php">

				<p> Nom ou pseudo : 
					<input type="text" name="nom" value=""> <span class="erreur">*</span>
				</p>

				<p> Courriel : 
					<input type="text" name="courriel" value=""> <span class="erreur"></span>
				</p>

				<p> Date (jj/mm/aaaa) : 
					<input type="text" name="date" value=""> <span class="erreur"></span>
				</p>
				<div class="questions">
					<p><strong>Question 1 :</strong> la norme de séparation radar est-elle respectée entre le SWW264 et le DIFCB&nbsp;?</p>
					<p><img class="pastropgrand" src="illustrations/sepradar1.png" alt="image d'une séparation radar"></p>
					<p><input type="radio" name="question1" value="oui">oui
						<input type="radio" name="question1" value="non">non</p>

					<p><strong>Question 2 :</strong> concernant le vol TRA85N</p>
					<p><img class="pastropgrand" src="illustrations/strip_tra.png" alt="image d'un strip"></p>
					<p><input type="radio" name="question2" value="transat">Son indicatif est «&nbsp;Air Transat 85 Novembre&nbsp;» (<em>'Novembre' est le nom de la lettre 'N' dans l'alphabet aéronautique)</em></p>
					<p><input type="radio" name="question2" value="B737">Ce vol s'effectue dans un Boeing 737-800</p>
					<p><input type="radio" name="question2" value="badod">Il est à destination de Badod, en Inde</p>

					<p><strong>Question 3 :</strong> le language utilisé par les pilotes et les contrôleurs pour communiquer entre eux s'appelle :</p>
					<p><input type="radio" name="question3" value="communication non violente">la communication non violente</p>
					<p><input type="radio" name="question3" value="standardisation">la standardisation</p>
					<p><input type="radio" name="question3" value="phraséologie">la phraséologie</p>
					<p><input type="radio" name="question3" value="protocole TCP">le protocole "transmission contrôleur-pilote"</p>\end{code2}
				\begin{code2}
					<p><strong>Question 4 :</strong> le vol EZY51NM a été autorisé à descendre vers </p>
					<p><img class="pastropgrand" src="illustrations/strip_ezy.png" alt="image d'un strip"></p>
					<p><input type="radio" name="question4" value="6360">une altitude de 6360ft</p>
					<p><input type="radio" name="question4" value="5000">une altitude de 5000ft</p>
					<p><input type="radio" name="question4" value="110">un niveau de vol 110 (environ 11000ft)</p>

					<p><strong>Question 5 :</strong> la norme de séparation radar est-elle respectée entre le DLH11P et le EZY34XF&nbsp;?</p>
					<p><img class="pastropgrand" src="illustrations/sepradar2.png" alt="image d'une séparation radar"></p>
					<p><input type="radio" name="question5" value="oui">oui
						<input type="radio" name="question5" value="non">non</p>

					<p><input type="submit" value="Vérifier mes réponses"></p>
				</div>
			</form>
		</section>
	</body>
\end{code2}

Code complet de la page de reception et traitement du quizz reception\_quizz.php :

\smallskip

\begin{code2}
<!DOCTYPE html>
<html lang="fr">
  	<head>
    	<title>Contrôleur aérien</title>
    	<meta charset="utf-8">
    	<link rel="stylesheet" type="text/css" href="style.css">
    	<meta name="description" content="Un quizz pour tester sa compréhension du contrôle aérien et de ses outils. Réalisé dans le cadre du cours 'Utilisation d'Ordinateurs en Réseau' de P. Kislin">
    	<meta name="url" content="atco.000webhostapp.com/receptionquizz.php">
    	<meta name="author" content="OD">
    	<meta name="keywords" content="contrôle aérien, contrôleur aérien, aiguilleurs du ciel, ICNA, ATCO, navigation aérienne, DGAC">
  	</head>
  	<body>
  		<h1>L'heure du quizz</h1>
  		<p><blockquote class="citation">Qualifié&nbsp;!</blockquote></p>\end{code2}
				\begin{code2}
<!--La barre de navigation-->
  		<ul class="navbar">
			<li class="l"><a href="index.html">Accueil JS</a></li>
			<li class="l"><a href="avecstyle.html">Accueil CSS</a></li>
			<li class="l"><a href="htmlnu.html">Accueil HTML</a></li>
			<li class="r"><a class="active"  href="quizz.html">Quizz</a></li>
		</ul>

<!--Traitement des réponses au questionnaire-->
		<?php 
			$nomErr = $courrielErr = $dateErr = "";
	  		$nom = $courriel = $date = "";
	  		$obl = "champs obligatoire";
	  		$nom_questions = array("question1", "question2", "question3", "question4", "question5"); // vecteur des noms des questions
	  		$rep_user = array(); // vecteur des réponses de l'utilisateur
	  		$rep_correctes = array("oui", "B737", "phraséologie", "5000", "oui"); // vecteur des réponses justes
	  		$resultats_user = array(); // vecteur de 0 ( si rep juste) et 1 (si faute)
	  		$questionnaire_ok = FALSE;

			/* fonctions qui accentuent la sécurité du formulaire (source : "https://www.w3schools.com/php/php_form_validation.asp" et le cours) */
	  		function nettoyer($data) {
	  			$data = trim($data); // enlève les caractères superflus
	  			$data = stripslashes($data); // enlève les "\"
	  			$data = htmlspecialchars($data); /* convertit les car. spéciaux en code html*/
	  			return $data;
			}
			/* Validation du format de la date, source : https://www.php.net/manual/en/function.checkdate.php */
			function validateDate($date, $format = 'd/m/Y') {
    			$d = DateTime::createFromFormat($format, $date);
    			return $d && $d->format($format) == $date;
			}
			/* Vérification de la validité des réponses, affichage des erreurs le cas échéant */
			// nom
			if (empty($_POST['nom'])) {
				$nomErr = $obl;
			} else {
				$nom = nettoyer($_POST['nom']);
				// vérifie si nom contient seulement lettres, chiffres et espaces
				if (!preg_match("/^[0-9a-zA-Z- Ééèô]*$/",$nom)) {
  				$nomErr = "seulement lettres, chiffres et espaces autorisés";
  				}
			} \end{code2}
				\begin{code2}
			// courriel
			if (!empty($_POST['courriel'])) {
				$courriel = nettoyer($_POST['courriel']);
				if ((!filter_var($courriel, FILTER_VALIDATE_EMAIL)) || (preg_match("/'/", $courriel))) {
  				$courrielErr = "adresse non valide";
				}
			} 

			// date - si date entrée, vérifie son format ; sinon date = date du jour
			date_default_timezone_set("Europe/Paris");
  			if(!empty($_POST['date'])) {
  				$date = $_POST['date'];
  				if (!validateDate($date)) {
  				$dateErr = "date non valide";
  				}
  			} else {
  				$date = date('d/m/Y');
  			}

  			// remplissage de l'arrays 'rep_user' avec les réponses de l'utilisateur, "" si pas répondu
  			for ($i = 0; $i < 5; $i++) {
  				if (!empty($_POST[$nom_questions[$i]])) {
  					array_push($rep_user, $_POST[$nom_questions[$i]]);
  				} else {
  					array_push($rep_user, "");
  				}
  			}

  			// remplissage de l'array 'resultats_user' avec 0 si rep fausse, 1 sinon
  			for ($i = 0; $i < 5; $i++) {
  				if ($rep_user[$i] == $rep_correctes[$i]) {
  					array_push($resultats_user, 1);
  				} else {
  					array_push($resultats_user, 0);
  				}
  			}

  			// score de l'utilisateur
  			$score_user = array_sum($resultats_user);

  			/* Vérification que nom, courriel et date sont bien remplis (pour affichage réponses) */
  			if ($nomErr == "" && $courrielErr == "" && $dateErr == "") {
  				$questionnaire_ok = TRUE;
  			}

  		?>\end{code2}
				\begin{code2}
<!--Accès à la base de données-->
		<?php

		/* MySQLi */
			if ($questionnaire_ok) {
				// connection bdd
				$conn = mysqli_connect("localhost", "id21044620_atco", "icna11b@Nice", "id21044620_atcodb");
				// test connection
				if (!$conn) {
					die("Connection failed : " . mysqli_connect_error());
				}

				// insertion réponses questionnaire dans bdd
				$sql_insert = "INSERT INTO id21044620_atcodb.questionnaire (`nom`, `courriel`, `date`, `question1`, `question2`, `question3`, `question4`, `question5`, `res1`, `res2`, `res3`, `res4`, `res5`) VALUES ('$nom', '$courriel', '$date', '$rep_user[0]', '$rep_user[1]', '$rep_user[2]', '$rep_user[3]', '$rep_user[4]', '$resultats_user[0]', '$resultats_user[1]', '$resultats_user[2]', '$resultats_user[3]', '$resultats_user[4]')";

				/* test insertion */
				if (mysqli_query($conn, $sql_insert)) {
					echo $nom . ", vos réponses ont bien été enregistrées.";
				} else {
					echo "Error : " . $sql_insert . "<br>" . mysqli_error($conn);
				} 

				// récupération du nombre de lignes dans la table
				$sql_nb_participants = "SELECT COUNT(`clef`) FROM id21044620_atcodb.questionnaire";
				$nb_participants = mysqli_fetch_row(mysqli_query($conn, $sql_nb_participants))[0];
			}
			
		?>

<!--Le questionnaire + affichage réponses-->
		<section>
			<p class="erreur">* champs obligatoire</p>
			<!--methode POST pour une question de sécurité-->
			<form method="post" action="https://atco.000webhostapp.com/reception_quizz.php">

				<p> Nom ou pseudo : 
					<input type="text" name="nom" value="<?php echo $nom;?>"> <span class="erreur">*<?php echo $nomErr;?></span>
				</p>\end{code2}
				\begin{code2}
				<p> Courriel : 
					<input type="text" name="courriel" value="<?php echo $courriel;?>"> <span class="erreur"><?php echo $courrielErr;?></span>
				</p>

				<p> Date (jj/mm/aaaa) : 
					<input type="text" name="date" value="<?php echo $date;?>"> <span class="erreur"><?php echo $dateErr;?></span>
				</p>
				<div class="questions">
					<p><strong>Question 1 :</strong> la norme de séparation radar est-elle respectée entre le SWW264 et le DIFCB&nbsp;?</p>
					<p><img class="pastropgrand" src="illustrations/sepradar1.png" alt="image d'une séparation radar"></p>
					<p><input type="radio" name="question1" <?php if (isset($rep_user[0]) && $rep_user[0]=="oui") echo "checked";?> value="oui">oui
						<input type="radio" name="question1" <?php if (isset($rep_user[0]) && $rep_user[0]=="non") echo "checked";?> value="non">non</p>

					<?php
						// affichage réponse si nom, courriel et date remplis correctement
						if ($questionnaire_ok) {
							// si bonne réponse
							if ($resultats_user[0]==1) {
								echo "<p class='jargon'>Bonne réponse ! ";
							} else {
								echo "<p class='erreur'>Perdu. ";
							}
							echo("Les deux avions sont à 2000ft (deuxième ligne de l'étiquette), mais ils sont séparés latéralement de 6,86NM. La norme est donc respectée.</p>");

							// affichage pourcentage bonne réponse à la question
							$sql_nb_res1_correct = "SELECT SUM(`res1`) FROM id21044620_atcodb.questionnaire";
							$nb_res1_correct = mysqli_fetch_row(mysqli_query($conn, $sql_nb_res1_correct))[0];
							echo "<p>" . round($nb_res1_correct/$nb_participants*100) . "&#37; des participants ont répondu correctement à cette question.</p>";
						}

					?>\end{code2}
				\begin{code2}
					<p><strong>Question 2 :</strong> concernant le vol TRA85N</p>
					<p><img class="pastropgrand" src="illustrations/strip_tra.png" alt="image d'un strip"></p>
					<p><input type="radio" name="question2" <?php if (isset($rep_user[1]) && $rep_user[1]=="transat") echo "checked";?> value="transat">Son indicatif est «&nbsp;Air Transat 85 Novembre&nbsp;» (<em>'Novembre' est le nom de la lettre 'N' dans l'alphabet aéronautique)</em></p>
					<p><input type="radio" name="question2" <?php if (isset($rep_user[1]) && $rep_user[1]=="B737") echo "checked";?> value="B737">Ce vol s'effectue dans un Boeing 737-800</p>
					<p><input type="radio" name="question2" <?php if (isset($rep_user[1]) && $rep_user[1]=="badod") echo "checked";?> value="badod">Il est à destination de Badod, en Inde</p>

					<?php
						if ($questionnaire_ok) {
							if ($resultats_user[1]==1) {
								echo "<p class='jargon'>Bonne réponse ! ";
							} else {
								echo "<p class='erreur'>Perdu. ";
							}
							echo("'B738' est le code pour un Boeing 737 de la série 800. Son indicatif est «&nbsp;Transavia 85 Novembre&nbsp;», et sa destination Amsterdam (code EHAM). Badod est simplement le nom d'un point sur sa route.</p>");
							// affichage pourcentage bonne réponse à la question
							$sql_nb_res2_correct = "SELECT SUM(`res2`) FROM id21044620_atcodb.questionnaire";
							$nb_res2_correct = mysqli_fetch_row(mysqli_query($conn, $sql_nb_res2_correct))[0];
							echo "<p>" . round($nb_res2_correct/$nb_participants*100) . "&#37; des participants ont répondu correctement à cette question.</p>";
						}
					?>
					<p><strong>Question 3 :</strong> le language utilisé par les pilotes et les contrôleurs pour communiquer entre eux s'appelle :</p>
					<p><input type="radio" name="question3" <?php if (isset($rep_user[2]) && $rep_user[2]=="communication non violente") echo "checked";?> value="communication non violente">la communication non violente</p>
					<p><input type="radio" name="question3" <?php if (isset($rep_user[2]) && $rep_user[2]=="standardisation") echo "checked";?> value="standardisation">la standardisation</p>
					<p><input type="radio" name="question3" <?php if (isset($rep_user[2]) && $rep_user[2]=="phraséologie") echo "checked";?> value="phraséologie">la phraséologie</p>
					<p><input type="radio" name="question3" <?php if (isset($rep_user[2]) && $rep_user[2]=="protocole TCP") echo "checked";?> value="protocole TCP">le protocole "transmission contrôleur-pilote"</p>\end{code2}
				\begin{code2}
					<?php
						if ($questionnaire_ok) {
							if ($resultats_user[2]==1) {
								echo "<p class='jargon'>Bonne réponse ! ";
							} else {
								echo "<p class='erreur'>Perdu. La bonne réponse est «&nbsp;la phraséologie&nbsp». ";
							}
							echo("Le protocole 'TCP' n'est pas d'une grande utilité dans ce contexte... La communication non violente, elle, gagnerait à être plus connue.</p>");

							// affichage pourcentage bonne réponse à la question
							$sql_nb_res3_correct = "SELECT SUM(`res3`) FROM id21044620_atcodb.questionnaire";
							$nb_res3_correct = mysqli_fetch_row(mysqli_query($conn, $sql_nb_res3_correct))[0];
							echo "<p>" . round($nb_res3_correct/$nb_participants*100) . "&#37; des participants ont répondu correctement à cette question.</p>";
						}
					?>

					<p><strong>Question 4 :</strong> le vol EZY51NM a été autorisé à descendre vers </p>
					<p><img class="pastropgrand" src="illustrations/strip_ezy.png" alt="image d'un strip"></p>
					<p><input type="radio" name="question4" <?php if (isset($rep_user[3]) && $rep_user[3]=="6360") echo "checked";?> value="6360">une altitude de 6360ft</p>
					<p><input type="radio" name="question4" <?php if (isset($rep_user[3]) && $rep_user[3]=="5000") echo "checked";?> value="5000">une altitude de 5000ft</p>
					<p><input type="radio" name="question4" <?php if (isset($rep_user[3]) && $rep_user[3]=="110") echo "checked";?> value="110">un niveau de vol 110 (environ 11000ft)</p>

					<?php
						if ($questionnaire_ok) {
							if ($resultats_user[3]==1) {
								echo "<p class='jargon'>Bonne réponse ! ";
							} else {
								echo "<p class='erreur'>Perdu. Il est autorisé à descendre à 5000ft. ";
							}
							echo("'6360' correspond à un code transpondeur, c'est un nombre détecté par le radar qui permet à ce dernier d'identifier le vol ; '110' est un cap magnétique (la direction à prendre par rapport à la rose des vents).</p>");\end{code2}
				\begin{code2}
							// affichage pourcentage bonne réponse à la question
							$sql_nb_res4_correct = "SELECT SUM(`res4`) FROM id21044620_atcodb.questionnaire";
							$nb_res4_correct = mysqli_fetch_row(mysqli_query($conn, $sql_nb_res4_correct))[0];
							echo "<p>" . round($nb_res4_correct/$nb_participants*100) . "&#37; des participants ont répondu correctement à cette question.</p>";
						}
					?>

					<p><strong>Question 5 :</strong> la norme de séparation radar est-elle respectée entre le DLH11P et le EZY34XF&nbsp;?</p>
					<p><img class="pastropgrand" src="illustrations/sepradar2.png" alt="image d'une séparation radar"></p>
					<p><input type="radio" name="question5" <?php if (isset($rep_user[4]) && $rep_user[4]=="oui") echo "checked";?> value="oui">oui
						<input type="radio" name="question5" <?php if (isset($rep_user[4]) && $rep_user[4]=="non") echo "checked";?> value="non">non</p>

					<?php
						if ($questionnaire_ok) {
							if ($resultats_user[4]==1) {
								echo "<p class='jargon'>Bonne réponse ! ";
							} else {
								echo "<p class='erreur'>Perdu. ";
							}
							echo("Les deux avions sont séparés latéralement de seulement 2,26NM, cependant, le premier est à 900ft, tandis que le deuxième est à un niveau 110 (environ 11000ft), plus de 10000ft les séparent. C'était un piège, en réalité on est large&nbsp;!</p>"); 

							// affichage pourcentage bonne réponse à la question
							$sql_nb_res5_correct = "SELECT SUM(`res5`) FROM id21044620_atcodb.questionnaire";
							$nb_res5_correct = mysqli_fetch_row(mysqli_query($conn, $sql_nb_res5_correct))[0];
							echo "<p>" . round($nb_res5_correct/$nb_participants*100) . "&#37; des participants ont répondu correctement à cette question.</p><p><strong>Votre score est de $score_user sur 5.</strong></p>
								<p>" . $nb_participants . " personnes ont participé jusqu'à présent.";
						} else {
							// Le bouton "Vérifier mes réponses" disparaît après la première tentative
							echo('<p><input type="submit" value="Vérifier mes réponses"></p>'); 
						}
					?>

				</div>
			</form>
		</section>
	</body>\end{code2}	$

\begin{figure}[h]
		\includegraphics[width=12cm, center]{CE/c5navigateur1.png}
		\caption{Chap 5 - CE du début du quizz dans firefox}
		\label{c5navigateur1}
		\end{figure}

\section{Annexes au chapitre 6}

Code complet de la page interactive index.html :

\smallskip

\begin{code2}
<!DOCTYPE html>
<html lang="fr">
  	<head>
    	<title>Contrôleur aérien</title>
    	<meta charset="utf-8">
    	<link rel="stylesheet" type="text/css" href="style.css">
    	<meta name="description" content="Une première approche du métier de contrôleur aérien. Réalisé dans le cadre du cours 'Utilisation d'Ordinateurs en Réseau' de P. Kislin">
    	<meta name="url" content="atco.000webhostapp.com/index.html">
    	<meta name="author" content="OD">
    	<meta name="keywords" content="contrôle aérien, contrôleur aérien, aiguilleurs du ciel, ICNA, ATCO, navigation aérienne, DGAC">

<!--code JS pour modifier titre et contenu de l'encart "zoom sur"-->
		<script>
			// les variables _txt correspondent au corps du paragraphe
			const phraseo_txt = 'La phraséologie est un langage proche de la langue parlée, mais qui utilise un <strong>vocabulaire restreint et des tournures de phrases imposées</strong>. C'est une langue vivante, qui évolue perpétuellement, au gré des incidents reportés par les différents acteurs de l'aéronautique. Un exemple simple : si un contrôleur dit à un pilote qui vole à 5000ft «&nbsp;descendez deux mille pieds&nbsp;», cela peut être compris de deux manières. La première, «&nbsp;descendez 2000ft&nbsp;», amènera le pilote à descendre à 2000ft ; la deuxième, «&nbsp;descendez de 1000ft&nbsp;», le fera descendre seulement à 4000ft. Pour lever le doute, lorsqu'il autorise un avion vers 2000ft, le contrôleur doit systématiquement utilisé le mot "altitude" : «&nbsp;descendez altitude 2000ft&nbsp;». La phraséologie existe en plusieurs langues ; en France, pilotes et contrôleurs peuvent parler français ou anglais entre eux. <strong>C'est le pilote qui impose sa langue</strong> lors du premier échange. En anglais, les mots "for" et "to" sont à proscrire dans la plupart des cas, puisqu'ils peuvent être compris "four" et "two". Derrière ces règles simples se cache un enjeu de sécurité important.';
			
			const sepradar_txt = '<img class="r" src="illustrations/sepradar_zoom2.png" alt="Une image radar">La norme de séparation se décline en deux conditions, dont l'une au moins doit être respectée : une <strong>séparation latérale de 3Nm minimum</strong> (environ 5,5km), et une <strong>séparartion verticale de 1000ft minimum</strong> (environ 300 mètres). L'espacement latérale minimale est relativement grand : c'est dû à l'imprécision des radars. Cette distance dépend en réalité des centres de contrôle, et peut monter jusqu'à 8Nm (presque 15km). L'espacement horizontal est plus précis, il est permis par les anémomètres à l'intérieur des avions, qui mesurent la pression extérieure et la convertissent en altitude. Sur l'image de droite, on voit les représentations de deux avions sur une image radar, avec pour chacun une <span class="jargon">étiquette</span> donnant des informations essentielles : en première\end{code2}
				\begin{code2} 			ligne, l'immatriculation ou l'indicatif de vol ; en deuxième ligne l'altitude ou le niveau de vol entre autres. Le DIFCB est à 2000ft, le AIZ731 à 3200ft en descente (petite flèche vers le bas à côté de l'altitude). S'il continue ainsi, la séparation verticale va bientôt passer en dessous de 1000ft&nbsp;! On vérifie alors la distance qui les sépare grâce au <span class="jargon">vecteur mesure</span> : la dernière ligne de l'encadré blanc montre que ces deux avions sont séparés de 3,04Nm. C'est un magnifique exemple d'un croisement fin.';
			
			const strip_txt = '<img class="pastropgrand" src="illustrations/strip.png" alt="La photo d'un strip">Un strip contient un grand nombre d'informations, souvent codifiée. L'espace de gauche donne les éléments généraux du vol. Entre autres, <strong>TVF7217</strong> (à prononcer «&nbsp;france soleil soixante-douze dix-sept&nbsp;») est l'<span class="jargon">indicatif du vol</span>&nbsp;; <strong>B738</strong> est le <span class="jargon">type d'avion</span>, un Boeing 737-800&nbsp;; <strong>LFTH</strong> et <strong>LFRS</strong> sont les <span class="jargon">codes <abbr title="Organisation de l'Aviation Civile Internationale">OACI</abbr></span> des aéroports de départ, Toulon-Hyères, et d'arrivée, Nantes-Atlantique. Un peu plus à droite, <strong>JULEE PADKO FJR</strong> sont les noms des points que le pilote a prévu de survoler (<span class="jargon">sa route</span>) en dessous desquels on peut également lire leurs heures de survol prévues. Encore à droite, les nombres <strong>2000 ... 170</strong> représentent les <span class="jargon">altitudes</span> ou <span class="jargon">niveaux de vol</span> que le contrôleur est susceptible d'autoriser.';

			// les variables _ttl correspondent au titre de l'encart
			const strip_ttl = '<larger> un strip </larger>';
			const phraseo_ttl = '<larger> la phraséologie </larger>';
			const sepradar_ttl = '<larger> la norme de séparation </larger>';

			/* les fonctions next et previous des deux boutons dépendent de ce qui est affiché dans le paragraphe principal */
			function next() {
				if (document.getElementById("par_zoom").innerHTML == strip_txt) {
					document.getElementById("par_zoom").innerHTML = sepradar_txt;
					document.getElementById("titre_zoom").innerHTML = sepradar_ttl;
				} else if (document.getElementById("par_zoom").innerHTML == sepradar_txt) {
					document.getElementById("par_zoom").innerHTML = phraseo_txt;
					document.getElementById("titre_zoom").innerHTML = phraseo_ttl;
				} else {
					document.getElementById("par_zoom").innerHTML = strip_txt;
					document.getElementById("titre_zoom").innerHTML = strip_ttl;
				}
			}\end{code2}
				\begin{code2} 
			function previous() {
				if (document.getElementById("par_zoom").innerHTML == strip_txt) {
					document.getElementById("par_zoom").innerHTML = phraseo_txt;
					document.getElementById("titre_zoom").innerHTML = phraseo_ttl;
				} else if (document.getElementById("par_zoom").innerHTML == sepradar_txt) {
					document.getElementById("par_zoom").innerHTML = strip_txt;
					document.getElementById("titre_zoom").innerHTML = strip_ttl;
				} else {
					document.getElementById("par_zoom").innerHTML = sepradar_txt;
					document.getElementById("titre_zoom").innerHTML = sepradar_ttl;
				}
			}
		</script>
  	</head>

  	<body>
    	<h1 title="Un peu plus près des étoiles">Métier : contrôleur aérien</h1>
    	<blockquote class="citation">Le contrôleur aérien, c'est celui qui contrôle les billets dans l'avion&nbsp;?</blockquote>
		<section>
<!--barre de navigation-->
			<ul class="navbar">
				<li class="l"><a class="active" href="index.html">Accueil JS</a></li>
				<li class="l"><a href="avecstyle.html">Accueil CSS</a></li>
				<li class="l"><a href="htmlnu.html">Accueil HTML</a></li>
				<li class="r"><a href="quizz.html">Quizz</a></li>
			</ul>
<!--corps du site-->
    		<h2 title="En bref">Le contrôle aérien dans les grandes lignes</h2>
    		<p>Les contrôleurs aériens, appelés parfois «&nbsp;aiguilleurs du ciel&nbsp;», travaillent soit dans un aéroport, soit dans un centre de contrôle en route. Ils sont en contact radio avec tous les avions pénétrant l'espace aérien dont ils ont la charge.</p>
    		<p>Les trois missions du contrôleur aérien sont&nbsp;:</p>
			<ol>
				<li><strong>Sécurité</strong> - bien entendu, la première mission du contrôleur aérien est d'assurer la sécurité des vols. Pour ce faire, il respecte des <span class="jargon">normes de séparation</span>. Par exemple, lors d'un décollage ou d'un atterrissage, il ne doit pas y avoir plus d'un avion sur la piste&nbsp;; en l'air, les aéronefs sont séparés de 1000 pieds verticalement (300 mètres) ou 3 miles nautiques latéralement (environ 5km).</li>
				<li><strong>Environnement</strong> - les contrôleurs aériens assurent ensuite le respect de contraintes environnementales, comme les hauteurs minimales de survol des agglomérations pour limiter les nuisances sonores.</li>
				<li><strong>Optimisation</strong> - le but de cette dernière mission est d'optimiser les trajectoires afin de réduire les temps de vol.</li>
			</ol>\end{code2}
				\begin{code2} 
			<h2 title="En pratique">Concrètement, comment ça marche&nbsp;?</h2>

			<h3>Sectorisation de l'espace aérien</h3>
			<p>L'espace aérien est divisé en volumes, appelés <span class="jargon">secteurs de contrôle</span>, qu'on peut se représenter comme des cubes pour simplifier. Chaque cube, et tous les avions qu'il contient, sont sous la responsabilité d'un binôme de contrôleurs. Lorsque le trafic aérien s'intensifie, ces cubes peuvent à leur tour être diviser en de plus petits cubes, afin de réduire la charge de travail des contrôleurs, ainsi que l'occupation de fréquence (voir Les outils du contrôleur aérien - la radio).</p>
			<p>Tout au long de son vol, un vol commercial est en contact radio avec les contrôleurs aériens, depuis la mise en route des moteurs jusqu'à leur extinction à l'arrivée. À chaque changement de secteur de contrôle (chaque fois qu'il passe d'un «&nbsp;cube&nbsp;» à un autre), il change de fréquence radio et contacte les contrôleurs du secteur suivant.</p>
			<figure>
				<img class="pastropgrand" src="illustrations/coupeVerticale.png" alt="Schéma simplifié d'un vol Marseille-Nice">
				<figcaption>Coupe Verticale des secteurs de contrôle traversés par un vol Marseille-Nice</figcaption>
			</figure>

			<h3>Les outils du contrôleur aérien</h3>
			<figure>
				<img class="pastropgrand" src="illustrations/positionControle2.png" alt="Une position de contrôle">
				<figcaption>Une position de contrôle dans la salle d'approche de Nice</figcaption>
			</figure>
			<p>Un binôme de contrôleurs aériens travaille sur une <span class="jargon"> position de contrôle</span>, une console regroupant tous les outils nécessaires : platine radio, écrans radars, <span class="jargon">tableau de strip</span>, platine téléphone et divers écrans d'information aéronautique. Les éléments essentiels sont :</p>
			<ul>
				<li><strong>La radio</strong></li>
			</ul>
			<p class="clearfix"><img class="l" src="illustrations/casque.png" alt="Une photo de casque"> Les échanges entre pilotes et contrôleurs se font uniquement par radio, un équipement inventé à la fin du XIX<sup>ème</sup> siècle et qui a peu évolué depuis&nbsp;. En particulier, une seule station peut émettre à la fois ; dans le cas contraire les émissions sont brouillées. Cela arrive dans le cas d'une trop grande occupation de fréquence : la fréquence est saturée, pilotes et contrôleurs n'arrivent plus à passer leurs messages. Pilotes et contrôleurs communiquent dans un langage standardisé, appelé <span class="jargon"> phraséologie</span>, dont le but est d'améliorer la compréhension. </p>
			<ul>
				<li><strong>L'image radar</strong></li>
			</ul>\end{code2}
				\begin{code2} 
			<p class="clearfix"><img class="r" src="illustrations/radar2.png" alt="Une image radar" height="230">L'image radar sert pour la surveillance et la séparation des aéronefs. Elle permet au contrôleur de se représenter le trafic aérien dans l'espace, et de maintenir la norme de séparation radar (3Nm/1000ft) entre tous les aéronefs.</p> 
			<ul>
				<li><strong>Les «&nbsp;strips&nbsp;»</strong></li>
			</ul>
			<p class="clearfix"><img class="l" src="illustrations/tableauStrip.png" alt="Une photo d'un tableau de strip" height="200"> Un <span class="jargon"> strip</span> est une bandelette de papier sur laquelle sont inscrites toutes les informations utiles sur un vol : indicatif du vol (le nom utilisé pour l'appeler), aéroports de départ et d'arrivée, trajectoire prévu, etc. À chaque instruction qu'il donne à un avion, le contrôleur met à jour le strip, par exemple en entourant l'altitude autorisée.</p>
		</section>

<!--encart "zoom sur"-->
		<div class="clearfix zoom">
			<ul class="pointless">
				<li class="l inline"><strong>Zoom sur...</strong></li>
				<li class="inline r"><button type="button" onclick="next()"> &gt;&gt;</button></li>
				<li id="titre_zoom" class="inline r"> un strip </li>
				<li class="inline r"><button type="button" onclick="previous()"> &lt;&lt;</button></li>
			</ul>
			<p id="par_zoom"></p>
		</div>
		<script>
				document.getElementById("par_zoom").innerHTML = strip_txt;
		</script>
		<hr>

<!--bas de page-->
		<div class="basdepage">
			<p>Ce site a été réalisé dans le cadre d'un TP, pour la première année de licence d'informatique de l'<abbr title="Institut d'Enseignement à Distance">IED</abbr> Paris 8.</p>
			<p>Vous voulez en savoir plus sur le contrôle aérien&nbsp;? Voici un <a href="https://devenir.controleuraerien.fr/">site</a> pas plus joli que le mien, sponsorisé par un syndicat.</p>
			<p>Un autre <a href="https://www.enac.fr/fr/mcta-controleur-aerien">lien</a> pour en apprendre plus sur la formation des contrôleurs, vers le site de l'<abbr title="École Nationale de l'Aviation Civile">ENAC</abbr>, seule école en France à former des contrôleurs aériens.</p>
		</div>
		
  	</body>
</html>
\end{code2}

		\begin{figure}[h]
		\includegraphics[width=14cm, center]{CE/c6navigateur2.png}
		\caption{Chap 6 - CE de l’encadré "Zoom sur" dans firefox (2/3)}
		\label{c6navigateur2}
		\end{figure}	
		
		\begin{figure}[h]
		\includegraphics[width=14cm, center]{CE/c6navigateur1.png}
		\caption{Chap 6 - CE de l’encadré "Zoom sur"dans firefox (3/3)}
		\label{c6navigateur1}
		\end{figure}	
\section{Annexes au chapitre 7}


\end{document}
